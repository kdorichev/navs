% $Date: 2008-10-30 17:47:17 +0300 (Thu, 30 Oct 2008) $
% $Rev: 32 $
% $Author: condor $
% $URL: file:///var/svn/repos/condor/latex/my-heart.tex $
%
\documentclass[12pt,article,a4paper,fittopage]{ncc}
\usepackage[resetfonts]{cmap}
\usepackage[utf8]{inputenc}
\usepackage[russian]{babel}
\ToCenter[h]{17cm}{25cm}
\usepackage{url,multicol}
\usepackage[plain,headings]{nccfancyhdr}

\newpagestyle{lheadings}[headings]{%
  \fancyhead[c]{\nouppercase{%
		\fancycenter{\slshape\rightmark}{\textit{<<Мое сердце~--- дом Христа>> --- Р.~Мунгер}}{\thepage}}}%
}

\pagestyle{lheadings}
\clubpenalty=10000
\widowpenalty=10000

\begin{document}

\titlepage
\title{МОЕ СЕРДЦЕ~--- ДОМ ХРИСТА}
\author{Роберт Бойд Мунгер}
\date{}
\maketitle
\thispagestyle{headings}
\setcounter{page}{1}
\begin{multicols}{2}
В послании Павла к~Ефесянам есть такие слова: <<Да даст вам, по богатству славы Своей, крепко утвердиться Духом Его во внутреннем человеке, верою вселиться Христу в~сердца ваши>>\Footnote{1}{Ефесянам 3:16-17}. Смысл этих слов может быть выражен так: <<Да вселится Христос~--- благодаря вере вашей~--- в~сердца и~да обретет в~них дом>>. 

Нет сомнения в~том, что одним из самых примечательных положений христианской доктрины является следующее: Сам Иисус Христос через присутствие в~нас Святого Духа входит в~человеческое сердце и~обитает в~нем, обретая там дом. Христос будет жить в~сердце любого пригласившего Его человека. 

Наш Господь сказал Своим ученикам: <<Кто любит Меня, тот соблюдет слово Мое; и~Отец Мой возлюбит его, и~Мы придем к~нему и~обитель у~него сотворим>>\Footnote{2}{Иоанна 14:23}. Однако Христос добавлял, что скоро Ему придется их оставить. Ученикам трудно было понять смысл сказанного. Как это Христос мог <<сотворить с~ними обитель>> и~в~то же время покидать их?

Любопытно, что перед этим наш Господь использует подобный же образ в~главе 14 Евангелия от Иоанна: <<И когда пойду и~приготовлю вам место, приду опять и~возьму вас к~Себе, чтоб и~вы были, где Я>>\Footnote{3}{Иоанна 14:3}. Он обещал Своим ученикам: как Он шел на небо, чтобы приготовить для них место и~однажды приветствовать их в~том месте, так и~у них есть теперь возможность приготовить для Него место в~сердцах своих, куда бы Он пришел и~где сотворил бы вместе с~ними обитель. Они этого не~понимали. Как это все могло произойти?

Но вот наступил день Пятидесятницы. Дух живого Христа снизошел на церковь, и~тогда ученики поняли, что предсказывал Господь. Бог не~обитал в~храме Ирода в~Иерусалиме! Бог не~обитал ни в~одном из рукотворных храмов. Но теперь, посредством чуда излившегося Духа, Бог станет обитать в~человеческих сердцах. Тело верующего станет храмом живого Бога, а~человеческое сердце~--- домом Иисуса Христа.

В день Пятидесятницы ученики за полчаса узнали об Иисусе больше, чем за последние три года. Мне трудно представить себе, чтобы человек мог быть удостоен более высокой чести, чем сотворение в~своем сердце дома для Христа, где Он мог бы приветствовать Его, служить Ему, угождать Ему, все лучше узнавать Его. 

Однажды вечером (я никогда не~забуду этот вечер) я~попросил Его войти в~мое сердце. И~Он принял мое приглашение! С~моей стороны не~последовало какого-либо всплеска эмоций, но все было очень реально, коснулось самых глубин моей души. Христос вошел в~тьму моего сердца и~включил свет. Он разжег огонь в~остывшем камине, и~холода не~стало. Там, где царила тишина, зазвучала музыка. На смену хаосу пришла гармония, пустоту моего сердца Он наполнил Своей исполненной любви чудесной дружбой. Мне ни разу не~пришлось пожалеть о~том, что я~открыл дверь Христу, и~я знаю, что никогда не~пожалею об этом впредь!

Все это, конечно, лишь первый шаг в~сотворении дома для Христа в~сердце. Он сказал: <<Се, стою у~двери и~стучу: если кто услышит голос Мой и~отворит дверь, войду к~нему и~буду вечерять с~ним и~он со Мною>>\Footnote{4}{Откровение 3:20}. Если вы хотите познать истинную сущность Иисуса Христа, живого Бога, сделать свое самое <<я>>~Его обителью, достаточно просто распахнуть дверь и~попросить Его войти и~стать вашим Спасителем и~Господом. После того как Христос вошел в~мое сердце, охваченный радостью по поводу новообретенных взаимоотношений, я~сказал Ему: <<Господь, я~хочу, чтобы мое сердце действительно принадлежало Тебе. Я~хочу, чтобы Ты поселился в~нем навсегда. Все, что у~меня есть, принадлежит Тебе. Позволь мне познакомить Тебя со всеми особенностями дома, который называется моим сердцем и~моей жизнью, чтобы Ты мог чувствовать Себя в~нем как можно полнее>>. Он был счастлив войти, и~счастлив, что для Него нашлось место в~моем заурядном сердце.

\pagestyle{lheadings}

\section*{Библиотека}

Первой комнатой в~доме был рабочий кабинет~--- библиотека. Давайте назовем его рабочим кабинетом нашего разума. В~моем доме кабинет разума небольшого размера, с~толстыми стенами. Но это важная комната. В~каком-то смысле в~ней осуществляется контроль над всем домом. Он вошел туда вместе со мной и~стал оглядываться вокруг, попеременно останавливая взгляд то на книгах, то на книжном шкафу, то на журналах на столике, то на картинах, висевших на стене. Следя за Его взглядом, я~чувствовал себя все более неловко. Странно, ведь прежде я~никогда не~видел в~них ничего плохого. Но теперь, когда мы вместе смотрели на эти вещи, я~все больше приходил в~замешательство. Его взор был слишком чист, чтобы останавливаться на некоторых из моих книг; немало хлама лежало на журнальном столике: христианину явно не~следовало тратить время на подобную “прессу”; что же до картин на стене, другими словами, до образов и~мыслей, занимавших мой разум, то за них мне стало просто стыдно.

Покраснев, я~повернулся к~Нему и~сказал: 

---~Господи, я~знаю, что эта комната нуждается в~большой уборке и~в~переделке. Не~поможешь ли Ты мне преобразовать ее, привести в~порядок?

---~Конечно,~--- ответил Он. ---~Я с~удовольствием помогу тебе. Для этого я~здесь и~нахожусь. Прежде всего, отбери из всего того, что ты читаешь и~на чем останавливается твой взгляд, бесполезное, не~служащее ни добру, ни истине, и~выброси прочь! Теперь поставь на освободившееся место книги Библии. Наполни свою библиотеку Писанием и~размышляй над ним день и~ночь. Что же до картин, то тебе нелегко будет совладать с~этими образами, но Я~смогу тебе помочь. 

И Он дал мне большой портрет Самого Себя. 

---~Повесь его в~центре комнаты,~--- сказал Он мне,~--- на стене своего разума. 

Я так и~сделал и~уже на протяжении нескольких лет прихожу к~выводу, что когда мои мысли сосредотачиваются вокруг Христа и~Его близости ко мне, плохие образы и~помыслы отступают под действием Его чистоты и~силы. Так Он Сам помог подчинить Себе мои мысли, но борьба с~дурными помыслами продолжается.

Позвольте мне призвать вас: если маленькая комната вашего разума оказывается для вас камнем преткновения~--- поселить в~ней Христа. Заполните ее словом Божьим, изучайте его и~постоянно размышляйте над ним в~присутствии Господа Иисуса Христа.

\section*{Столовая}

Из библиотеки мы направились в~столовую~--- комнату моих вкусов и~желаний. Это была очень большая, важнейшая для меня комната, где я~проводил немало времени, старательно добиваясь удовлетворения всех своих потребностей.

Я сказал Ему: 

---~Это моя самая любимая комната, и~я совершенно уверен, что все, подаваемое здесь, придется Тебе по вкусу.

Он сел за стол и~спросил: 

---~А что сегодня на ужин?

---~Ну,~--- ответил я,~--- мои любимые блюда: деньги, университетские дипломы, акции с~гарниром из газетных статей о~славе и~удаче. 

Вся эта мирская пища мне действительно нравилась. Ничего принципиально дурного в~каждом из перечисленных <<блюд>> не~было. Однако, для настоящего духовного питания эта пища не~годилась. Когда она была поставлена перед Ним, Он ничего не~сказал. Но я~заметил, что Он ничего не~ел, и~тогда, несколько обеспокоенный, я~сам обратился к~Нему: 

---~Спаситель, Ты и~внимания не~обращаешь на эту еду. В~чем дело?

---~Я питаюсь пищей, которая тебе не~известна,~--- ответил Он.~---~Эта пища~--- исполнение воли Пославшего Меня. Он снова взглянул на меня и~сказал: 

---~Если ты хочешь пищи, которая действительно насыщает, ищи воли Отца, а~не собственных удовольствий и~удовлетворения своих желаний. Стремись порадовать Господа, и~такая пища принесет тебе удовлетворение. Попробуй вот это!

И тогда же, за столом, Он дал мне почувствовать вкус исполнения Божьей воли. Ничто в~мире не~могло сравниться со вкусом этой пищи. Только она приносит насыщение и~удовлетворение. Все прочее от голода не~избавляет.

Каково меню в~столовой наших желаний? Чем мы угощаем нашего Божественного друга, и~что едим сами? <<\ldots{}всё, что в~мире: похоть плоти, похоть очей и~гордость житейская\ldots{}>>\Footnote{5}{1 Иоанна 2:16}, наши эгоистичные желания? Или для повседневной трапезы своей вы выбираете Божью волю?

\section*{Гостиная}

Затем мы вошли в~гостиную. Эта комната была тихой и~уютной. Я~ее любил. Здесь находились камин, мягкие кресла, книжный шкаф, диван, а~главное~--- спокойная, теплая атмосфера. 

Ему как будто бы тоже понравилась наша гостиная. Он сказал: <<Это действительно чудесная комната. Давай почаще приходить сюда. Здесь царят уединение и~покой, и~все располагает к~беседам и~общению>>.

Ну что же, будучи еще молодым христианином, я, конечно, затрепетал. Мне и~в~голову не~могло прийти, что я~мог бы предпочесть что-либо иное нескольким минутам, проводимым наедине со Христом в~сердечном общении с~Ним. 

Он пообещал: <<Я буду приходить сюда каждый день рано поутру. Давай здесь и~будем встречаться и~вместе начинать день>>. И~вот у~меня вошло в~обыкновение каждое утро спускаться в~гостиную, куда приходил и~Он. Иисус вынимал и~книжного шкафа Библию, открывал ее, и~мы вместе принимались за чтение. Он рассказывал мне о~ее сокровищах и~раскрывал истины, заключавшиеся в~ней. Он рассказывал мне о~спасительной истине, открывающейся с~ее страниц. Моя душа пела от радости, когда Он говорил о~том, что Он сделал для меня и~кем Он для меня станет. Эти часы, проведенные вместе, были поистине чудесными. Посредством Библии и~Святого Духа Иисус обращался ко мне. Я~отвечал Ему в~молитве. Так, во время уединения и~личного общения со Христом, крепла моя дружба с~Ним.

Но мало-помалу под давлением многочисленных обстоятельств время это стало сокращаться. Я~не знаю, почему так получалось, просто мне казалось, что я~слишком занят, чтобы регулярно уделять время Христу. Понимаете, это было не~намеренно, а~случалось как-то само собой. В~конце концов сократилось не~только время, проводимое нами вместе, я~даже стал пропускать дни: то по причине экзаменов в~университете, то в~силу каких-то иных срочных дел, постоянно вытеснявших время уединения и~бесед с~Иисусом. Бывало и~так, что я~пропускал два дня подряд и~более.

Помню, однажды утром я~в~спешке направлялся на какую-то важную встречу. Пробегая мимо гостиной, я~обратил внимание на то, что дверь приоткрыта. Заглянув, я~увидел огонь в~камине и~Господа, сидящего у~огня. В~огорчении я~вдруг подумал про себя: <<Он~--- мой гость. Я~пригласил Его в~свое сердце. Он вошел туда как Спаситель и~Друг, чтобы жить со мною>>. И~вот я~повернулся и~нерешительно вошел в~гостиную. Сокрушенно взглянув на Него, я~проговорил: 

---~Благой Господь, прости меня. Ты приходил сюда каждое утро?

---~Да,~--- ответил Он.~--- Я~ведь сказал тебе, что буду приходить каждое утро, чтобы встречаться с~тобой. 

Теперь я~уже не~знал, куда деваться от стыда. Он остался верным, несмотря на мою неверность. Я~попросил у~Него прощения, и~Он простил меня. Так Он поступает всегда, когда мы искренне признаём свои проступки и~хотим исправиться. Он сказал: 

---~Твоя беда вот в~чем: ты думал о~времени уединения, посвященном изучению Библии, и~о молитве, как о~чем-то, имеющем значение для твоего духовного роста. Это верно, но ты забыл, что часы, проведенные вместе, и~для Меня кое-что значат. Помни: Я~люблю тебя, Я~уплатил высокую цену за твое искупление. Я~хочу дружбы с~тобой. И~когда ты обращаешь свой взор ко Мне, Мое сердце уже радуется. Так что,~--- продолжал Он,~--- не~пренебрегай этим утренним временем хотя бы только ради Меня. Какое бы желание ни владело тобой, помни, что Я~дорожу нашей дружбой и~по-настоящему люблю тебя!

И, знаете ли, сознание того, что Христос Сам хочет моей дружбы, что Он любит меня, хочет, чтобы я~оставался с~Ним, ждет меня, чтобы быть со мной, изменило мое отношение ко времени, проводимому наедине с~Богом, более чем что-либо еще. Не заставляйте же Христа подолгу и~понапрасну ожидать вас в~гостиной вашего сердца, но используйте каждый день, чтобы, читая слово Божье и~обращаясь к~Нему в~молитве, поддерживать с~Ним общение.

\pagestyle{lheadings}
\section*{Мастерская}

Вскоре Он спросил меня: <<Есть ли в~твоем доме мастерская?>> Она находилась в~подвале дома моего сердца. Там стоял верстак и~были сложены кое-какие инструменты. Но я~редко работал с~ними. Иногда я~спускался вниз и~делал небольшие вещицы. Однако ничего существенного у~меня никогда не~получалось.

Я повел Его вниз. Он оглядел верстак, оценив мои скромные таланты и~умения, и~затем сказал: 

---~Технически здесь все предусмотрено. Что же ты делаешь в~своей жизни для царства Божьего? 

Взглянув на две-три маленькие игрушки, валявшиеся на верстаке, Он поднес одну из них ко мне. 

---~И такие вот небольшие безделушки~--- это все, что ты делаешь для других в~своей христианской жизни?

Я места не~мог найти от смущения. 

---~Господи,~--- сказал я,~--- это лучшее, что я~в~состоянии делать! Я~понимаю, что это немного, и~мне стыдно признавать, что на большее мне не~хватает ни умения, ни сил.

---~Хотел бы ты действовать успешнее?~--- спросил Он.

---~Конечно!~--- ответил я.

---~Хорошо. Но сперва вспомни, чему Я~научил тебя: <<\ldots{}без Меня не~можете делать ничего>>\Footnote{6}{Иоанна 15:5}. Теперь расслабься и~позволь Моему Духу действовать через тебя. Обратись ко Мне. Я~знаю, что ты неумел и~неловок, но Дух~--- отличный мастер, и~если Он возьмет под контроль твои руки и~сердце, то Сам станет действовать через тебя. Теперь повернись. 

И вот, возложив Свои великие, сильные руки на мои и~направляя работу моих инструментов Своими искусными пальцами, Он начал действовать через меня. 

---~Расслабься. Ты все еще пребываешь в~напряжении. Дай мне поработать Самому!

Меня изумляет, что Его умелые руки творят через мои руки, когда я~полностью доверяю Ему и~полагаюсь на Него.

Я еще очень многому должен научиться и~еще очень далек от удовлетворенности тем, что у~меня получается, по-прежнему иногда препятствую Ему. Но я~знаю: что бы ни производилось для Бога, производится во мне Его сильной рукой и~силой Его Духа, во мне обитающего.

Так что не~огорчайтесь, если не~можете ничего сделать для Бога. Главное не~в~наших способностях, а~в нашей готовности. Предавайте себя Христу. Чутко реагируйте на Его желания. Доверяйтесь Ему, и~Он совершит через вас такие дела, которые изумят вас самих.

\section*{Комната досуга и~развлечений}

Вспоминаю то время, когда Он спросил меня о~комнате, в~которой я~провожу досуг. Я~все надеялся, что Он не~задаст мне этого вопроса, потому что кое-какие из моих дружеских связей и~кое-что из моих развлечений я~предпочел бы сохранить при себе. Я~понимал, что Христу они не~могут понравиться, Он их не~одобрит, и~поэтому всячески избегал этого вопроса.

Но вот как-то вечером я~решил отправиться с~приятелем в~ночную прогулку по городу и, когда я~уже был на пороге, Он взглядом остановил меня.

---~Ты уходишь?

---~Да,~--- ответил я.

---~Хорошо, Я~бы хотел пойти с~тобой,~--- сказал Он.

---~Ну,~--- смущенно ответил я,~--- я~не думаю, что Тебе в~самом деле понравится там, куда мы идем. Вот завтра вечером~--- другое дело. Завтра мы вместе пойдем на занятие по изучению Библии или на молитвенное собрание, но сегодня я~встречаюсь с~совсем другими людьми.

---~Как хочешь,~--- сказал Он.~--- Я, правда, думал, когда пришел в~твой дом, что мы будем все делать вместе. Ведь мы хотели стать хорошими друзьями. Во всяком случае знай: Я~хочу пойти с~тобой.

---~Ладно,~--- сказал я,~--- завтра мы пойдем куда-нибудь вместе.

В тот вечер я~чувствовал себя жалким и~несчастным. <<Какой же я~друг Христу,~--- размышлял я,~--- если покидаю Его ради того, чтобы делать такие вещи и~ходить в~такие места, которые, я~знаю наверняка, Ему бы не~понравились?>> Вернувшись в~тот вечер домой, я~заметил свет в~Его комнате и~вошел, чтобы поговорить с~Ним. 

---~Господь,~--- сказал я,~--- сегодня я~получил урок. Теперь я~знаю, что без Тебя не~могу хорошо проводить время. С~сегодняшнего дня мы все будем делать вместе.

Мы спустились в~ту комнату, где я~проводил свой досуг, и~Он преобразил ее. Он привнес в~мою жизнь истинную радость, подлинное счастье, нерушимую дружбу. Смех и~музыка звучат с~тех пор в~доме моего сердца. Он улыбнулся, и~глаза Его блеснули:

---~Ты думал, что когда Я~рядом, повеселиться не~удастся? Помни, Я~пришел, чтобы Моя радость пребывала в~тебе и~чтобы твоя радость была совершенна\Footnote{7}{Иоанна 15:11}.

\section*{Спальня}

Однажды, когда мы были в~спальной комнате, Он спросил меня, что за фотография висит у~моей кровати.

---~Это фотография моей подружки,~--- ответил я. Полагая, что в~наших с~ней отношениях нет ничего дурного, я~все же неловко улыбнулся, говоря о~них Христу. В~некоторых вопросах мы с~ней никак не~могли прийти к~согласию, и~я не~хотел обсуждать их с~Христом. Я~попытался сменить тему разговора. Но Иисус, должно быть, узнал, о~чем я~думал. 

---~Ты начинаешь сомневаться, правильно ли то, что Я~говорю о~сексе, не~так ли? Правильно ли, что вступать в~половую связь друг с~другом должны лишь те, кто соединен узами брака? Тебе кажется, что Я~прошу от тебя чего-то неестественного, если невозможного. Ты боишься, что следуя в~этом Моей воле, ты не~сможешь получать полное удовольствие от жизни и~от любви. Верно?

---~Да,~--- признался я.

---~Тогда выслушай внимательно то, что Я~тебе сейчас скажу,~--- продолжал Он. ---~Я запрещаю прелюбодеяние и~внебрачные половые связи не~потому, что секс~--- это плохо, а~потому, что секс~--- это хорошо. Будучи физическим удовольствием, он также является звеном, соединяющим два сердца в~глубокой любви. Это мощная созидательная сила, лежащая в~основе человеческой жизни. Секс приобретает громадный положительный потенциал, если относиться к~нему надлежащим образом. При неправильном же отношении секс обращается во зло. Вот почему Бог желает, чтобы секс находил свое выражение лишь в~рамках приверженности узам, связывающим жизни людей непреходящей любовью. Позволь Мне оказать тебе помощь во взаимоотношениях с~прекрасным полом. Если ты почувствуешь себя виноватым, знай, что Я~по-прежнему люблю тебя и~не покидаю тебя. Поговори со Мной об этом! Признай, что поступил дурно! Предпринимай меры, чтобы этого больше не~повторилось! Полагайся на Мою силу, дабы не~пасть и~вступить в~исполненные любви брачные отношения, когда двое воистину становятся одна плоть во Мне.

\pagestyle{lheadings}
\section*{Потаенный чуланчик}

Мне хотелось бы рассказать вам еще об одном событии, имевшем грандиозное значение. Однажды я~увидел, что Христос ожидает меня в~дверях. Он внимательно смотрел на меня и, когда я~вошел, сказал: 

---~В доме какой-то странный запах. Пахнет гнильем. Там наверху. Видимо, запах идет из того потаенного чуланчика. 

Не успел Он договрть, ак я~уже знал, о~чем идет речь. Да, там на лестничной площадке действительно находился маленький чулан, в~несколько квадратных метров, и~в~нем я~хранил под замком кое-какие личные вещи. Я~не хотел, чтобы кто-нибудь их увидел, тем более Христос. Я~знал, что это прогнившая мертвечина, и~все-таки я~любил и~так хотел сохранить их, что даже самому себе боялся признаться в~их существовании. Вместе с~Ним я~стал неохотно подниматься по лестнице, и~по мере того, как мы приближались к~чулану, запах становился сильнее и~сильнее. Он указал на дверь и~сказал:

---~Вот здесь! Вот откуда идет этот гнилостный запах!

Я разозлился. Только так я~могу определить мое тогдашнее состояние. Как! Я~допустил Его в~библиотеку, в~столовую, в~гостиную, в~мастерскую, в~комнату развлечений, а~теперь Он еще спрашивает меня и~об этом крошечном чулане! 

---~Это слишком,~--- промолвил я~про себя. ---~Этот ключ я~Ему не~отдам.

---~Хорошо,~--- откликнулся Он, прочитав мои мысли. ---~Если ты думаешь, что Я~останусь на втором этаже и~буду вдыхать этот запах, то ошибаешься. Я~переберусь подальше, в~заднюю часть дома или еще куда-нибудь, потому что не~собираюсь с~этим мириться. 

И я~увидел, как Он начал спускаться вниз.

Если однажды вы узнали и~полюбили Христа, то худшее, что вы можете испытать~--- это ощущение того, что вы теряете Его дружбу, и~Он отдаляется от вас. Я~вынужден был сдаться. 

---~Я отдам Тебе этот ключ,~--- проговорил я~печально. ---~Но Тебе придется Самому отпереть чулан и~вычистить его. У~меня нет на это сил.

---~Я знаю, что у~тебя нет сил,~--- сказал Он. ---~Тебе надо лишь отдать Мне ключ, лишь предоставить Мне право заняться этим чуланом. И~Я займусь им. 

Дрожащими пальцами я~передал Ему ключ. Он взял его у~меня, направился к~двери, открыл ее, вошел, вытащил гнившие там отбросы и~вышвырнул их прочь. Потом Он вычистил чулан, заново перестроил и~выкрасил его~--- все это в~мгновение ока. Тут же свежий, душистый ветер ворвался в~дом. Воздух стал совершенно другим. О, что это за победа, что за освобождение~--- очистить жизнь от гнили! Какой бы грех, какая бы горечь ни омрачали прошлое, Иисус всегда готов простить, исцелить и~восполнить утраченное.



\section*{Дом перешел к~иному владельцу}

Затем новая мысль осенила меня. Ради Христа я~пытался блюсти свое сердце в~чистоте, готовности, но это так трудно делать. Я~начинал с~одной комнаты, но не~успевал привести ее в~порядок, как другая становилась грязной. Пока я~очищал от мусора второй этаж, первый вновь покрывался пылью. Я~так устаю, стараясь содержать свое сердце в~чистоте. И~жить новой жизнью, исполненной послушания, мне просто не~по силам! И~тогда я~задал вопрос: 

---~Господь, могу ли я~надеяться, что Ты примешь на Себя ответственность за весь дом и~станешь трудиться в~нем для меня, словом, действовать так, как в~том чулане? Могу ли я~возложить на Тебя ответственность за содержание моего сердца в~должном состоянии и~просить, чтобы Ты направлял мою жизнь куда следует?

Я увидел как Его лицо просияло. 

---~С радостью,~--- ответил Он,~--- ведь для этого Я~и~пришел. Опираясь на собственные силы, ты не~сможешь жить христианской жизнью. Это невозможно. Позволь Мне сделать это через тебя и~для тебя. Только такой способ действенен. Но,~--- медленно продолжал Он,~--- Я~ведь не~являюсь владельцем этого дома. Я~здесь только гость и~не~имею права распоряжаться собственностью, которая Мне не~принадлежит.

Мгновенно поняв Его, я~с~волнением воскликнул: 

---~Господь, Ты был моим гостем, а~я пытался выступать в~роли хозяина. Но с~этого момента я~становлюсь в~этом доме слугой. Хозяином же и~владельцем будешь Ты. 

Со всех ног я~бросился к~сейфу, достал из него документы, удостоверяющие право на владение домом и~содержащие опись всего моего имущества, а~также перечисляющие обязанности владельца и~характеризующие состояние и~особенности дома. Затем, вернувшись к~Нему, я~с~радостью переписал всю эту собственность на Его имя на вечные времена. Упав на колени, я~передал документы Ему. 

---~Вот,~--- сказал я,~--- здесь все, что я~сам и~что есть у~меня. Отныне и~всегда владей этим домом. Я~же лишь остаюсь здесь с~Тобой как слуга и~друг.

В тот день Он взял на Себя ответственность за мою жизнь, и~я даю вам слово, что лучшего способа жить истинно христианской жизнью нет. Он знает, как управлять ею, на что ее употреблять. И~глубокий мир снизошел с~тех пор на мою душу. 

Пусть же Господь Христос поселится и~в~доме вашего сердца как Господин всего, что вы имеете.
\thispagestyle{lheadings}
\end{multicols}

\vspace{0.5cm}
\hrule
\vspace{0.3cm}
\begin{footnotesize}
\noindent
Originally published as \textsc{``My heart---Christ's home''} by Robert Boyd Munger.
Copyright \copyright{} 1954 InterVarsity Christian Fellow\- ship/USA. Translated and
distributed with permission from InterVarsity Press, P.O. Box 1400,
Downers Grove, IL 60515, USA. \url{www.ivpress.com}

\vspace{0.2cm}
\noindent
Перевод статьи: Христианское издательство <<Триада>>, 101000, Москва, а/я 371, \url{info@triad.ru}, 
\linebreak  \url{www.triad.ru},  +7~495~963-54-79, +7~495~964-85-89.

\vspace{0.2cm}
\noindent
Цитаты Библии в~тексте выполнены по Синодальному переводу.

\vspace{0.2cm}
\noindent
Редактирование и~верстка: Навигаторы, Россия; \url{navigators.ru@gmail.com}

\end{footnotesize}
{\tiny 
\begin{verbatim}
$LastChangedDate: 2008-10-30 17:47:17 +0300 (Thu, 30 Oct 2008) $
$Revision: 32 $
\end{verbatim}

}

\end{document}
