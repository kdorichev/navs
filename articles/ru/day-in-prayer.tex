% $Date: 2008-10-31 15:08:28 +0300 (Fri, 31 Oct 2008) $
% $Rev: 36 $
% $Author: condor $
% $URL: file:///var/svn/repos/condor/latex/day-in-prayer.tex $
%
\documentclass[12pt,article,a4paper,fittopage]{ncc}
\usepackage[resetfonts]{cmap}
\usepackage[utf8]{inputenc}
\usepackage[russian]{babel}
%\ToCenter[h]{17cm}{25cm}
\FromMargins[h]{2cm}{2cm}{1cm}{1cm}
\usepackage{url,multicol}

\usepackage[plain,headings]{nccfancyhdr}
\newpagestyle{lheadings}[headings]{%
	\fancyhead[c]{\nouppercase{%
		\fancycenter{\slshape\rightmark}{\textit{<<Как провести день в~молитве>>~--- Л.~Сэнни}}{\thepage}}}%
}
\pagestyle{lheadings}
\clubpenalty=10000
\widowpenalty=10000

\pdfinfo{
   /Author (Lorne C. Sanny)
   /Title  (How to Spend the Day in Prayer)
   /Subject (Prayer)
%  /URL (http://lifestream.org/pdf/RuHLM.pdf)
   /Keywords (Russian translation)
   /Producer (kdorichev@gmail.com)
}

\begin{document}

\begin{center}
\includegraphics{lorne_sanny}

\thispagestyle{empty}

\vspace{0.5cm}
\textsc{\LARGE{КАК ПРОВЕСТИ ДЕНЬ В МОЛИТВЕ}}\\[0.3cm]
Лорн \textsc{Сэнни}
\end{center}

\begin{flushright}
\textit{<<Воспользуйтесь этой величайшей честью, ниспосланной вам с~небес. Иисус Христос умер ради того, чтобы такая связь и~общение с~Отцом стали возможны>>}
\par
---~Билли~Грэм
\end{flushright}

\begin{flushright}
\textit{<<Молитва~--- могучая вещь, потому что Бог связал себя обязательствами с~ней>>}
\par
---~Мартин~Лютер
\end{flushright}

\begin{flushright}
\textit{<<Знакомство с~Богом не~происходит на скорую руку. Он не~расточает даров Своих первому встречному или тому, кто торопится прийти и~уйти. Продолжительное общение с~Ним~--- ключ к~познанию Его и~обретению влияния>>}	
\par
---~Эдвард~М.~Баундс
\end{flushright}

\begin{multicols}{2}
<<Я никогда не~думал, что один день может так изменить меня,~--- сказал мне однажды мой друг.~---~Мне кажется, я~стал лучше относиться к~людям и~к~тому, что меня окружает. Почему я~не~делаю этого чаще?>> 

Подобное можно услышать от людей, проведших день в~молитве. При такой занятости, когда многие важные для нас вещи требуют нашего времени, настоящая, не~наспех произнесенная молитва кажется для нас скорее роскошью, чем необходимостью. Тем более~--- провести целый день в~молитве!

Библия предлагает нам три руководящих принципа относительно молитвы. Первый: <<Непрестанно молитесь>>~--- дух молитвы приведет вас в~такую гармонию с~Богом, что в~любое время дня вы сумеете возвысить свои сердца в~молитве и~прославлении Господа.

Второй принцип~--- это практика уединения с~Богом или практика утреннего бодрствования. Мы наблюдаем ее в~жизни Давида\Footnote{1}{Псалом 5:3}, Даниила\Footnote{2}{Даниила 6:10} и~самого Господа\Footnote{3}{Марка 1:35}. Это время, ежедневно отводимое для размышления над словом Божиим и~молитвы, необходимо для духовного здоровья и~роста христианина.

В Писании имеются примеры и~того (это~--- третий принцип), когда довольно длительное время посвящается только молитве: Иисус проводил в~молитве ночи напролет; Неемия, услышав о~тяжелом положении Иерусалима, молился целыми днями; Моисей три раза проводил наедине с~Богом сорок дней и~ночей.

\section*{Учиться у~Бога}

Именно в~те необычайные дни молитвы Бог открыл Моисею Свои пути и~замыслы\Footnote{4}{Псалом 102:7}. Он позволил Моисею как бы заглянуть в~щелочку в~заборе, даровал ему особое прозрение, тогда как рядовые израильтяне лишь наблюдали деяния Божии, ежедневно разворачивающиеся у~них на глазах.

Однажды я~сказал одному своему знакомому Доусону Тротману, что он производит впечатление человека, который знает какое у~него предназначение~--- быть орудием в~руках Божиих. <<Я не~думаю, что это так,~--- ответил он,~--- но вот что я~знаю: Бог дал мне несколько  обещаний, и~я знаю, что Он выполнит их>>. В~предшествующие годы Доусон регулярно подолгу проводил время наедине с~Богом. Да и~в~моей жизни продолжительные молитвы в~парке Сиэтла, на холмах в~южной Калифорнии или в~Божественном саду в~Колорадо Спрингс освежали и~укрепляли меня, помогая понять, в~чем заключается для меня воля Божья. 

Такие часы, посвященные молитве, могут стать отправными точками и~в~вашей жизни. Ежедневное уединение с~Богом окажется более плодотворным, если вы будете больше молиться, прислушиваясь к~тому, что говорит вам Господь. Время, отведенное для сосредоточенных размышлений, подготовит вас к~«непрестанной» молитве, побуждая провести весь день в~общении с~Богом.

Возможно, вы никогда до этого не~молились подолгу, так как не~видели в~этом необходимости или вообще плохо представляете себе, как это делается.

\pagestyle{lheadings}
\section*{Зачем нужен день молитвы}

Для чего нам при всей нашей занятости тратить столько времени на молитву?

\begin{itemize}
\item \textbf{Ради более продолжительного общения с~Богом.} Ради пребывания с~Ним, размышления о~Нем. Бог призвал нас к~общению с~Сыном Своим, Иисусом Христом\Footnote{5}{1 Коринфянам 1:9}. Как и~любые личные отношения между людьми, эта дружба крепнет от времени, проведенного вместе. Иисус не~забывает такие встречи с~вами.  <<Внимает Господь и~слышит это, и~пред лицем Его пишется памятная книга о~боящихся Господа и~чтущих имя Его>>\Footnote{6}{Малахии 3:16}.
\item \textbf{Для обновления перспективы.} Посвящая целый день молитве, мы взлетаем, как птицы, над землей и~получаем возможность мысленно обозреть этот мир с~точки зрения Бога. Испытывая трудности, мы особо нуждаемся в~такой перспективе для того, чтобы увидеть невидимое и~расставить конкретные, осязаемые вещи на свои места. Наши средства духовной защиты становятся надежнее по мере того, как <<мы смотрим не~на видимое, но на невидимое: ибо видимое временно, а~невидимое~--- вечно>>\Footnote{7}{2 Коринфянам 4:18}.
\item \textbf{Для ходатайствования и~заступничества.} У~каждого из нас есть друзья и~родственники~--- не~христиане, которых мы хотели бы привести к~Господу, есть соседи, служители в~церкви. Хорошо известна сила молитвы и~ее влияние на людей и~на ход событий, однако многие христиане не~часто прибегают к~молитве в~этих целях. Между тем время, в~которое мы живем, становится все тревожней, в~свете чего нам необходимо обращать молитву как на какие-то добрые свершения, так и~на препятствование злу.
\item \textbf{Для осмысления собственной жизни перед лицом Господа}, для подведения итогов и~личной самооценки. Особенно важно так подолгу, целый день молиться перед принятием каких-то важных решений. Но, конечно же, наиболее плодотворно молиться таким образом регулярно. В~такой день вы сможете оценить насколько далеко вы находитесь от своих целей и~получить наставления Господа через Его Слово. В~Библии содержатся обещания, данные нам с~вами, подобные тем, что были даны Хадсону Тейлору, Джорджу Мюллеру или Доусону Тротману. И~именно тогда, когда мы остаемся наедине с~Богом, Он дарует нам внутреннюю уверенность в~том, что Он осуществит Свои обещания.
\item \textbf{Для соответствующей подготовки.} Яркий пример тому~--- Неемия. Проведя несколько дней в~непрестанной молитве и~взывая к~Господу, Неемия был призван к~царю. <<И~сказал мне царь: чего же ты желаешь? Я~помолился Богу небесному и~сказал царю: если царю благоугодно\ldots>> И~далее он поведал царю о~своем замысле\Footnote{8}{Неемии 2:4--5}. А~потом~--- <<Встал я~ночью с~немногими людьми, бывшими при мне, и~никому не~сказал, что Бог мой положил мне на сердце сделать для Иерусалима>>\Footnote{9}{Неемии 2:12}. Когда Бог положил ему это на сердце? Очевидно, именно в~те часы и~дни, когда Неемия постился, молился и~ожидал ответа от Него. Так что когда пришло время действовать, он был уже готов.
\end{itemize}

Однажды маленький мальчик спросил у~одного летчика, долго ли ему придется обдумывать, где он посадит самолет в~случае каких-нибудь неполадок. Пилот ответил, что нет и~что он прекрасно знает, где и~как ему посадить свой самолет в~случае необходимости, так как он уже все обдумал заранее.

То же должно происходить и~в~нашей христианской жизни. Если в~часы нашего уединения с~Богом Он открывает нам Свои замыслы, то в~нужный момент мы готовы будем последовать за Ним. Нам не~придется говорить: <<Мы не~готовы>>. Причина того, что многие христиане порой упускают свои возможности, состоит не~в том, что их сознание не~готово, а~в том, что они не~подготовлены в~сердце своем. Подготовка же эта происходит тогда, когда мы остаемся наедине с~Богом.

\section*{Молитесь, исходя из Слова Божьего}

<<В первый год царствования его я, Даниил, сообразил по книгам число лет, о~котором было слово Господне к~Иеремии пророку, что семьдесят лет исполнятся над опустошением Иерусалима. И~обратил я~лице мое к~Господу Богу с~молитвою и~молением, в~посте и~вретище и~пепле, и~молился я~Господу Богу моему, и~исповедывался\ldots>>\Footnote{10}{Даниила 9:2--4}.

Итак, Даниил <<сообразил>> по Писанию о~предстоящих событиях. И~поскольку слово Божье открылось ему, он начал молиться. Сказано, что Бог имеет замыслы и~намерения, следовательно, Он дает обещания. Поэтому мы можем сказать, что мы молимся об обещанном, чтобы замыслы Господа осуществились. Так, у~Господа был замысел совершить что-то, и~Он дал в~отношении этого обещание, поэтому Даниил и~молился. Таким образом, Даниилу предстояло завершить некую цепь, подобно цепи с~электрическим током, в~которой и~проявляется сила электричества.

Провести день в~молитвах к~Господу~--- не~значит усесться где-нибудь на камне в~позе роденовского <<Мыслителя>> и~задуматься над первой попавшейся мыслью, какая придет вам в~голову. Нет. В~этот день вам надо распахнуть свою душу перед Словом Божьим, и~настанет момент, когда оно само поведет вас в~молитве. Если вы целиком углубитесь в~самокопание, думая лишь о~себе и~своих проблемах, ничего хорошего не~выйдет. В~конце концов, в~расчет принимается не~ваше мнение о~самом себе. Оценивает Бог. И~результат этой оценки Он откроет вам через Библию, послав вам в~помощь Духа Святого. Его Слово поведет вас в~молитве.

\section*{Как это сделать?}

Отведите целый день или часть дня для молитвы. Возьмите с~собой немного еды, воды и~выйдите из дому. Найдите уединенное место, где бы ничто не~отвлекало вашего внимания. Постарайтесь не~углубляться в~любование прелестями окружающей вас природы~--- не~тратьте время зря. Поймав себя на том, что наблюдаете а~белками и~муравьями, переведите свое внимание в~должное русло, прочитав Псалом 103 Давида, который настроит вас на размышления о~том, как проявляется Божья сила в~Его творении.

\pagestyle{lheadings}

Возьмите с~собой Библию, записную книжку и~карандаш, сборник песен/гимнов, может быть, какую-нибудь духовную литературу. Я, например, в~таких случаях люблю просмотреть брошюру Э.М.~Браундса <<Сила через молитву>>\Footnote{11}{Power Through Prayer, by Edward M.~Bounds} или <<Слова к Ловцам Душ>> Горация Бонара\Footnote{12}{Words to Winners of Souls, by Horatius Bonar}, либо биографию какого-нибудь миссионера, например, <<За горными хребтами>> Ж. Тейлор\Footnote{13}{Behind the Ranges, by Geraldine Taylor}, где рассказывается о~победах, достигнутых Джеймсом О. Фрейзером в~Китае через его молитвы и~проповеди.

Даже если в~вашем распоряжении целый день, используйте его как можно продуктивнее и~не~теряйте времени с~самого начала.

Поделите этот день на три части: ожидание Господа, молитва за других и~молитва за себя.

\subsection*{1. Ожидайте Господа}

Ожидая Господа, не~спешите. Не~следует ожидать каких-то мистических или экстатических переживаний. Просто ищите Господа своей душой и ожидайте Его. Есть множество мест в~Писании говорящих об ожидании Господа. <<Господи! рано услышь голос мой,~--- рано предстану пред Тобою и~буду ожидать>>\Footnote{14}{Псалом 5:4}.  И~еще: «Душа моя ожидает Господа более, нежели стражи~--- утра, более, нежели стражи~--- утра>>\Footnote{15}{Псалом 129:6}.

Ожидайте Его прежде всего для того, чтобы осознать Его присутствие. Вчитайтесь в~отрывок, подобный Псалму 138, и~вы проникнетесь ощущением, что Бог рядом. Поразмышляйте о~том, что нет места во всей Вселенной, где бы Его не~было. Нередко мы подобны Иакову, сказавшему: <<Истинно Господь присутствует на месте сем, а~я не~знал!>>\Footnote{16}{Бытие 28:16}.

Ожидайте Его, также и~для того, чтобы  очиститься. Два последних стиха Псалма 138 подводят нас к~этой мысли. Просите Господа, как говорится в~этих стихах, испытать ваше сердце. Если же мы сами будем пытаться делать это, мы легко можем оказаться во власти своего воображения, нездорового анализа, можем оказаться незащищенными перед действиями врага. Но когда сердца наши испытывает Дух Святой, Он привлекает ваше внимание к~тому, в~чем следует исповедаться, чтобы очиститься. Псалмы 50 и~31~---  это песни покаяния Давида. Они помогут вам. Утвердитесь на прочном основании, которое дает 1-е~Послание Иоанна 1:9, и~уповайте на верность Бога обещанию прощать нам всякий грех, который мы исповедуем.

Если вы осознаете, что согрешили против брата своего, то запланируйте исправление ошибки, иначе вам трудно будет общаться с~Богом. Он не~станет говорить с~вами, если между вами и~тем человеком все еще стоит нечто, над чем вы не~запланировали потрудиться при первой же возможности.

Ожидая Господа, просите Его дать вам способность как следует сосредоточиться, не~позволять себе заниматься фантазированием.

\pagestyle{lheadings}

Ожидайте Господа, чтобы прославить Его. Псалмы 102, 110 и~144~--- прекрасные примеры того, как можно воздать хвалу величию и~могуществу Господа Бога нашего. Собственно, б\'{о}льшая часть Псалмов~--- это и~есть молитвы. Мы можем также обратиться к~Откровению, главам 4 и~5, и~также их использовать для вознесения чести и~славы Богу. Нет лучшего способа молитвы, чем молитва по Писанию.

Если вы взяли с~собой сборник песен/гимнов, то можете петь Господу. Сколько написано чудесных гимнов, которые передают в~словах то, что мы сами не~в состоянии выразить. Возможно, вы не~слишком хорошо поете, ну так убедитесь, что никто из посторонних вас не~слышит, и~с радостью восхвалите Господа! Поверьте, для Него это ценно.

Так, естественным образом, вы перейдете к~благодарению. Подумайте обо всем том замечательном, что Бог сделал для вас, и~поблагодарите Его за все: за спасение вашей души, за те благословения, которые Он излил на вас, за вашу семью и~друзей, за благоприятные возможности, предоставленные вам. Конечно же, за что-то вы благодарите Бога ежедневно, но теперь выразите Ему свою признательность вообще, за все бесчисленные блага, ниспосланные вам.

\subsection*{2. Молитесь за других}

Теперь подошло время неспешной подробной молитвы за тех, кого вы обычно в~своих молитвах не~упоминаете. Присоедините их к~тем, за кого вы молитесь постоянно. Обогните весь земной шар, молясь за людей в~других странах.

Вот три предложения относительно того, за что следует молиться.
\begin{enumerate}
\item Вначале просите об исполнении особых нужд людей. Возможно, вы помните их, либо записали со слов тех, кто вам известен. Молитесь о~нуждах, изложенных в~письмах миссионеров. Молитесь за то, чтобы людям были даны духовные сила, мужество, физическая выносливость, ясность ума и~т.д. Представьте себя на месте тех, за кого вы молитесь, и~просите Господа соответственно их ситуации.

\item Затем просмотрите молитвы, представленные в~Писании. Просите о~том же, о~чем просил Павел для других людей в~первой главе Посланий Филиппийцам и~Колоссянам, а~также в~первой и~третьей главах Послания Ефесянам. Это поможет вам продвинуться в~вашей молитве дальше, чем просто <<Господи, благослови тех-то и~тех-то и~помоги им так-то и~так-то>>.

\item Наконец, просите для других о~том же, о~чем просите для себя. Желайте им того, что Господь подсказывает вам.
\end{enumerate}

\pagestyle{lheadings}

Если вы молитесь о~ком-либо словами из какого-то стиха или обещания из Писания, вы можете сделать соответствующую пометку на своем молитвенном листе под именем этого человека, чтобы воспользоваться тем же стихом, когда будете молиться за него в~другой раз. Не забудьте потом этот стих, чтобы поблагодарить Господа, когда получите ответ на эту молитву.

\subsection*{3. Молитесь за себя}

Третью часть дня отведите молитве за самого себя. Если вы накануне принятия какого-нибудь важного решения, помолитесь об этом прежде, чем начнете молиться о~других.

В~своей молитве опирайтесь на Писание и~просите Господа, чтобы Он даровал вам понимание~--- в~соответствии с~Псалмом 118:18. Поразмышляйте над стихами из Библии, которые вы выучили, или над обещаниями, о~которых говорит Библия. Можно прочитать одну из книг полностью, от начала до конца; лучше вслух. Подумайте, как бы вы могли применить только что прочитанное в~вашей жизни.

Пример, достойный подражания, содержится в~1~Паралипоменон 4:10. Иавис молился так: 

\begin{quote}
<<О, если бы Ты благословил меня Твоим благословением, распространил пределы мои, и~рука Твоя была со мною, охраняя меня от зла, чтоб я~не~горевал!>>
\end{quote}

 Эта молитва~--- о~вашей жизни, о~вашем процветании, о~присутствии Божьем в~вашем сердце и~в~ваших делах, и~о Его защите. Иавис молился согласно воле Божьей, и~Бог ниспослал ему, чего он просил.

Основной настрой третьей части этого дня молитвы должен быть: 
\begin{center}
<<Господи, что Ты думаешь о~моей жизни?>>
\end{center}
 Рассмотрите свои цели и~устремления в~свете Божьей воли в~отношении вас. Иисус сказал: <<Моя пища есть творить волю Пославшего Меня и~совершать дело Его>>\Footnote{17}{Иоанна 4:34}. Хотите ли вы исполнять Божью волю больше, чем свою?

Рассмотрите ваши дела и~поступки в~свете поставленных перед вами задач. Бог может говорить вам об изменениях в~вашем распорядке дня и~в~планах, об отказе от некоторых видов деятельности, которые, возможно, сами по себе неплохи, но вовсе не~способствуют и~даже, скорее, препятствуют вашему успеху. Вы можете почувствовать, что ваша привычка проводить определенным образом вечера или выходные дни не~одобряется, так как, действуя иначе, вы могли бы проводить их с~большей пользой и~при этом отдыхать ничуть не~хуже.

Пока мол\'{и}тесь, записывайте свои мысли по поводу вашего поведения, использования времени, обдумывайте, как изменить их к~лучшему. Возможно, вы поймете, что необходимо лучше готовиться к~занятиям или вы по-другому отнесетесь к~какой-нибудь личной встрече; быть может, Господь вдохновит вас сделать для кого-нибудь нечто особенное и~т.~д. Записывайте всё.

В~своей молитве поделитесь с~Господом вашими планами на будущее, вашими устремлениями и~надеждами, решениями, которые вам необходимо принять. Приведите ваши доводы в~пользу того или иного решения, еще раз тщательно обдумайте их, молясь об этом и~сверяясь с~Писанием. В~тех местах Библии, которые вы прочитали в~течение этого дня, вы сможете найти соответствующие советы или указания. После молитвы вы наверняка обретете твердую определенность на основании ваших выводов. В~этом, собственно, и~заключается цель таких дней молитвы. Однако, вы должны быть готовы и~к тому, что некоторые проблемы так и~останутся без ответа. Но не~отчаивайтесь. Возможно, Бог просто считает, что еще не~время. Быть может, вы вскоре обнаружите, что ваша истинная нужда состояла не~в том, чтобы узнать свой следующий шаг, а~в каком-то новом откровении от Бога. 

Не стоит самим неловко и~неумно бросаться на поиски каких-то новых откровений. Сосредоточьтесь на тех обещаниях, которые вам уже известны. Если вы уже выучили наизусть библейские стихи из <<Тематической системы запоминания>>, то начните с~размышления над стихами из раздела <<Уверенность в~водительстве Божьем>>. Снова и~снова возвращайтесь к~тем прежним обещаниям. Молитесь о~том, чтобы они стали частью и~вашей жизни.

Переосмысляя давно известные мне обещания, я~сам получил немало величайших благословений от Господа. Они являются как бы нашей путеводной нитью, держась за которую, мы приходим к~другим. Библия полна их.

Вы можете отметить или подчеркнуть в~своей Библии некоторые обещания, которые дает Господь, написать дату и~пару слов об этом.
Очень важно, чтобы день молитвы не~был монотонным и~однообразным. Молитву следует чередовать с~чтением и~прогулкой. Один мой знакомый, например, молясь, ходит по комнате. Вместо того, чтобы скрючиться в~одной какой-то позе, распрямитесь, время от времени немного походите; словом, некое разнообразие будет небесполезно. 

По мере того, как у~вас будут возникать новые мысли, старайтесь включать их в~свою молитву. Если они касаются каких-то важных для вас дел, записывайте их. Это вполне естественно, что на протяжении дня молитвы на ум приходит то, что должно было быть сделано, так что записывайте все, что возникает у~вас в~голове. Молитесь о~том, что является предметом этих соображений, думайте, как осуществить это. Не пытайтесь отделаться ни от одной возникающей идеи, иначе мысль о~ней не~даст вам покоя до самого вечера.

В конце дня подведите в~вашей записной книжке итог: о~чем же Бог говорил сегодня с~вами? Вам будет очень полезно вновь вернуться к~своим записям спустя некоторое время.

\section*{Два вопроса}

Результатом вашего дня молитвы должны быть ответы на два вопроса, с~которым Павел обратился к~Господу по дороге в~Дамаск\Footnote{18}{Деяния 22:6--10}. Первый вопрос был: <<Кто Ты, Господи?>> Господь ответил: <<Я~--- Иисус>>. И~ваша задача состоит в~том, чтобы познать Его, узнать кто Он. Второй вопрос был: <<Господи! что мне делать?>> Господь дал ответ, предназначающийся конкретно Павлу. И~вот, в~те часы, когда вы вдумчиво и~неспеша станете искать Его волю, вы получите ответ на этот вопрос либо утвердитесь в~том, что уже знаете.

Не ожидайте, что этот день для вас закончится каким-то новым откровением или исключительным ощущением. Ожидайте Бога и~откройте сердце свое для принятия слова Его. Стремление пережить нечто необычайное (особенно для того, чтобы потом кому-нибудь рассказать) лишь собьет вас с~правильного пути. Порой какое-то новое переживание отвлекает наше внимание от насущного. Успешным ли оказался ваш день молитвы можно судить не~по тому, насколько вы оживлены или возбуждены к~его окончанию, но по тому, как это время подействует на вас, начиная со следующего дня. Если вы действительно раскрыли свое сердце Слову Божьему и~услышали Его, то это непременно скажется на вашей дальнейшей жизни.

День молитвы не~приходит сам собой. Помимо попыток врага рода человеческого~--- сатаны~--- отвлечь нас от молитвы, в~окружающем мире нас ожидает множество дел. Так что нам надо специально заранее запланировать и~выделить день, полностью посвященный общению с~Богом: например, пусть это будет первое число каждого месяца или раз в~три--четыре месяца. Бог даст вам на это Свое благословение~--- сделайте это скорее! Может быть, потом и~вы спросите себя: <<А~почему бы не~делать этого чаще?>>
\end{multicols}

\thispagestyle{lheadings} 

\vspace{0.5cm}
\hrule
\vspace{0.3cm}
\begin{footnotesize}

\noindent
Originally published as \textsc{``How to Spend the Day in Prayer''} by Lorne C. Sanny.

\vspace{0.2cm}
\noindent
Цитаты Библии в~тексте выполнены по Синодальному переводу.

\vspace{0.2cm}
\noindent
Перевод статьи: Христианское издательство <<Триада>>, 101000, Москва, а/я 371, \url{info@triad.ru}, 
\linebreak \url{www.triad.ru},  +7~495~963-54-79, +7~495~964-85-89.

\vspace{0.2cm}
\noindent
Редактирование и~верстка: Навигаторы, Россия; \url{navigators.ru@gmail.com}
\end{footnotesize}

{\tiny 
\begin{verbatim}
$LastChangedDate: 2008-10-31 15:08:28 +0300 (Fri, 31 Oct 2008) $
$Revision: 36 $
\end{verbatim}
}
\end{document}
