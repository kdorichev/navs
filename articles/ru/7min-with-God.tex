% $Date: 2008-10-31
% Updated: 2024-03-04
% $Prepared by: kdorichev@gmail.com
% $URL: https://github.com/kdorichev/navs/blob/master/articles/7min-with-God.tex
%
\documentclass[11pt,article,a4paper,fittopage,oneside]{article}
\usepackage{cmap}					          % поиск в PDF
\usepackage[T2A]{fontenc}			      % кодировка
\usepackage[utf8]{inputenc}			    % кодировка исходного текста
\usepackage[english,russian]{babel}	% локализация и переносы
\usepackage{indentfirst}
\frenchspacing

\usepackage{geometry} % Простой способ задавать поля
    		\geometry{top=20mm}
    		\geometry{bottom=10mm}
    		\geometry{left=15mm}
    		\geometry{right=15mm}

\usepackage{url,multicol}
\usepackage[plain,headings]{nccfancyhdr}
\usepackage{hyperref}

\newpagestyle{lheadings}[headings]{%
	\fancyhead[c]{\nouppercase{%
		\fancycenter{\slshape\rightmark}{\textit{<<Семь минут с Богом>>~--- Р.~Фостер}}{\thepage}}}%
}
\pagestyle{lheadings}
\clubpenalty=10000
\widowpenalty=10000

\pdfinfo{
   /Author (Robert D. Foster)
   /Title  (Seven minutes with God. How to plan a daily quiet time.)
   /Subject (Quiet time, Prayer, Fellowship with God.)
   /Keywords (Russian translation)
   /Producer (kdorichev@gmail.com)
}

\begin{document}
\begin{center}
\textsc{\LARGE СЕМЬ МИНУТ С БОГОМ}\\[0.5 cm]
\textsc{\Large Как планировать ежедневное время общения с~Богом}\\[0.5cm]
\emph{Автор:}\ Роберт \textsc{Фостер}
\end{center}

\setcounter{page}{1}
\thispagestyle{empty}

\begin{multicols}{2}

Это произошло в 1882-м году в~стенах Кембриджского университета. Именно тогда мир впервые услышал призыв:
\begin{center}
<<Помните утренние часы!>>
\end{center}
Cтуденты Хупер и~Торнтон заметили, что их дни перегружены занятиями, лекциями, играми и~проч. Жизнь была полна энтузиазма и~деятельности. Вскоре эти преданные Господу парни обнаружили трещину в~своих духовных доспехах, которая, если не~была бы залатана, могла привести к~катастрофе.

Они стали искать решение проблемы и~вскоре разработали схему, которую назвали <<Утренние часы>>. Ее суть заключалась в~том, чтобы проводить первые минуты нового дня наедине с~Богом, молясь и~читая Библию.

Утренние часы залатали трещину, освежили истину, которая часто забывается под напором непрекращающейся деятельности:

\begin{center}
\textbf{Чтобы знать Бога, необходимо постоянно проводить с~Ним время.}
\end{center}

Эта идея зажгла пламя. Последовал замечательный период Божьего благословения, который достиг кульминации, когда из богатых и~образованных парней образовалась т.~н. Кембриджская семерка. Они пожертвовали всем, чтобы стать миссионерами и~поехать в~Китай во имя Христа.

Но эти парни обнаружили, что вставать по утрам раньше, чтобы пообщаться с~Богом, было так же трудно, как и~жизненно необходимо. Торнтон постарался обуздать леность дисциплиной. Он изобрел автоматическое, безотказное средство от лености. Это было устройство, которое устанавливалось рядом с~его кроватью и~работало так: будильник приводил в~движение рыболовные снасти, и~простынь, прикрепленная к~леске, медленно сползала с~тела спящего. \textbf{Торнтон стремился встретиться с Богом!}

Близость единения со Христом должна быть снова обретена через утреннее время общения с~Ним. Называйте это как хотите~--- Тихое время, Утренние часы, Время уединения с~Богом или Личное поклонение~--- эти святые минуты в~начале каждого дня объясняют скрытый секрет христианства. Это та золотая нить, которая связывает вместе всех великих мужей Божьих: от Моисея до Давида Ливингстона, от пророка Амоса до Билли Грэма~--- богатого и~бедного, бизнесмена и~военнослужащего. У~каждого человека, который когда-либо добивался успехов для Бога, приоритетом \textnumero{}1~--- время уединенного общения с Богом!

Давид писал в~Псалме 56:8, <<Готово сердце мое, Боже, готово сердце мое>>. Подготовленное и~утвержденное сердце придает жизни стабильность. Не у~многих христиан такие сердце и~жизнь. Одним из отсутствующих звеньев является действенный план того, как начать и~поддерживать утреннее общение с~Господом.

Для того, чтобы ощутить благотворность личного общения с~Господом, я~хочу предложить вам начать с~семи минут. Возможно вы назовете это <<ежедневной семиминуткой>>. Пяти минут может быть недостаточно, а~десять для некоторых будет слишком много для начала.

Желаете ли вы посвятить семь минут каждого утра общению с~Господом? Не пять раз в~неделю, не~шесть дней из семи, а~все семь дней! Просите Бога помочь вам: \begin{quote}
<<Господи, я~хочу общаться с~Тобою первым по утрам по крайней мере семь минут. Завтра, когда будильник прозвенит в~6:15, у~меня назначена с~Тобой встреча>>.
\end{quote}

%\pagestyle{lheadings}
Вашей молитвой может быть:

\begin{quote}
<<Господи, рано услышь голос мой, рано предстану пред Тобою, и~буду ожидать>> (Псалом 5:4).
\end{quote}

Как вам провести эти семь минут? Встав с~кровати и~совершив утренний туалет, найдите уединенное место и~там, вместе с~Библией, насладитесь семью минутами общения с~Господом.

Посвятите первые 30 секунд молитве, подготавливая свое сердце. Поблагодарите Его за возможность хорошо выспаться и~о~предстоящем дне. \begin{quote}<<Господь, очисти мое сердце, чтобы я~мог услышать Твой голос, обращенный ко мне через Писание. Открой мое сердце и~наполни его. Сделай мой ум бдительным, мою душу активной, мое сердце восприимчивым. Господь, окружи меня Своим присутствием в~эти минуты. Аминь>>.\end{quote}

Теперь уделите несколько минут чтению Библии. Ваша величайшая нужда состоит в~том, чтобы услышать слова от Бога. Позвольте Божьему Слову зажечь огонь в~вашем сердце. Повстречайтесь с~Автором!

Рекомендую начать чтение с~одного из Евангелий, например, с~Евангелия от Марка. Читайте последовательно~--- стих за стихом, главу за главой. Не~торопитесь, но в~то же время избегайте остановок, чтобы выполнить исследование определенного слова, мысли или теологической проблемы. Читайте просто для удовольствия и~чтобы дать Богу возможность обратиться к~вам через Его Слово, может быть всего лишь 20~стихов, а~может и~всю главу. Когда закончите Марка, начните Евангелие от Иоанна. Вскоре вам захочется вернуться и~прочесть весь Новый Завет.

После того как Бог обратится к~вам через Свое Слово, ответьте Ему в~молитве. Теперь у~вас осталось две с~половиной минуты, чтобы пообщаться с~Богом, используя все или некоторые из перечисленных ниже аспектов молитвы:
\begin{enumerate}
\item \textbf{Прославление.}  Это чистейший вид молитвы, потому что она полностью адресована Господу. В~ней ничего нет лично для вас. Вы не~вламываетесь невежливо в~Божье присутствие, а~входите подобающим образом, вознося хвалу и~салютуя Царю царей. Итак, поклонитесь Ему, расскажите Господу, как вы Его любите. Отразите в~молитве Его величие, Его силу, Его суверенность, Его справедливость.
\item \textbf{Исповедание.} Повстречавшись с~Ним, вы захотите быть уверенным, что все ваши грехи очищены и~прощены. Значение слова <<исповедание>> происходит от однокоренного слова, означающего <<прийти к согласию, согласиться>>. Примените это в~молитве. То есть, придите к~согласию с~Богом. Что-то, случившееся вчера, вы называете небольшим искажением действительности~--- Бог называет это ложью! Вы называете это крепкими словцами~--- Бог называет это злословием. Вы называете это легким флиртом, увлечением~--- Бог называет это прелюбодеянием. <<Если бы я~видел беззаконие в~сердце моем, то не~услышал бы меня Господь>> (Псалом 65:18).
\item \textbf{Благодарение.} Выразите свою благодарность Богу. Подумайте о~нескольких конкретных вещах, за которые вы хотели бы Его поблагодарить: семья, работа, церковь, ваше служение. Благодарите Его даже за трудности. <<За все благодарите: ибо такова о~вас воля Божия во Христе Иисусе>> (1 Фесс. 5:18).
\item \textbf{Прошение.} Это означает <<просить искренне и смиренно>>. Это та часть вашей молитвенной жизни, в~которой вы открываете свои просьбы Богу. Просите за других, затем за себя. Почему бы не~включить в~ваш молитвенный список людей из других стран, миссионеров, студентов, обучающихся за границей, друзей, находящихся далеко, а~также тех людей по всему миру, которые еще не~слышали благой вести о~Христе.
\end{enumerate}

Давайте подведем итог:
\vspace{0.3cm}

\begin{tabular}{ r p{5.4cm} }
  0.5\ мин. & Молитва о~водительстве Божьем (Пс.142:8)\\
    4\ мин. & Чтение Библии (Пс. 118:18) \\
  2.5\ мин. & Молитва 
\end{tabular}

\begin{itemize}
\item прославление (1 Пар. 29:11);
\item исповедание (1 Ин. 1:9); 
\item благодарение (Ефес. 5:20);
\item прошение (Матф. 7:7).
\end{itemize}

Это просто руководство. Вскоре вы обнаружите для себя, что невозможно проводить только семь минут в~общении с~Господом. Произойдет изумительная вещь~--- семь минут превратятся в~двадцать, и~не пройдет много времени, как вы будете проводить в~общении с~Господом тридцать драгоценных минут.

Не становитесь преданными привычке, но будьте преданными Господу. Не делайте этого только потому, что другие делают. Пусть это не~будет бездуховной ежеутренней обязанностью. Применяйте это, потому что сам Бог дал нам бесценную привилегию общения с~Ним. Заключите с~Господом завет, что вы будете соблюдать и~развивать ваше общение с~Ним. 

\end{multicols}

\hrule
\begin{footnotesize}
\noindent
\begin{flushleft}
Статья является общественным достоянием (public domain). Originally published in English as  \textsc{``Seven minutes with God. How to plan a~daily quiet time.''} by Robert D.~Foster, NavPress, \href{https://www.navpress.com/}{www.navpress.com}.
\end{flushleft}

\noindent
Перевод, редактирование и~верстка: Навигаторы, Россия.

\noindent
Цитаты Библии в~тексте выполнены по Синодальному переводу.

\end{footnotesize}
\end{document}
