% $Date: 2008-11-19 10:57:13 +0300 (Срд, 19 Ноя 2008) $
% $Rev: 42 $
% $Author: condor $
% $URL: file:///var/svn/repos/condor/latex/tyranny.tex $
%
\documentclass[12pt,article,a4paper,fittopage]{ncc}
\usepackage[resetfonts]{cmap}
\usepackage[utf-8]{inputenc}
\usepackage[russian]{babel}
%\ToCenter[h]{17cm}{25cm}
\FromMargins[h]{2cm}{2cm}{1cm}{1cm}
\usepackage{url,multicol}
\usepackage[plain,headings]{nccfancyhdr}

\newpagestyle{lheadings}[headings]{%
  \fancyhead[c]{\nouppercase{%
		\fancycenter{\slshape\rightmark}{\textit{<<Под гнетом срочных дел>>~--- Ч.~Хаммель}}{\thepage}}}%
}

\pdfinfo{
   /Author (Charles E. Hummel)
   /Title  (Tyranny of the Urgent)
%   /Subject (Prayer)
%  /URL (http://lifestream.org/pdf/RuHLM.pdf)
   /Keywords (Russian translation)
   /Producer (kdorichev@gmail.com)
}

\pagestyle{lheadings}

\clubpenalty=10000
\widowpenalty=10000

\begin{document}

\begin{center}
\includegraphics{HummelCharles}

\thispagestyle{empty}

\vspace{0.5cm}
\textsc{\LARGE{ПОД ГНЕТОМ СРОЧНЫХ ДЕЛ}}\\[0.3cm]
Чарльз \textsc{Хаммель}
\end{center}

\setcounter{page}{1}
\begin{multicols}{2}
Не мечтали ли вы когда-нибудь о~тридцатичасовых сутках? Дополнительное время сняло бы то тяжелое напряжение, в~котором мы живем. Наша жизнь оставляет за собой широкий шлейф неоконченных дел: письма, на которые мы не~ответили; друзья, которых мы не~навестили; статьи, которые мы не~написали; книги, которые мы не~прочитали\ldots{} Мысли о~них преследуют нас, стоит нам остановиться на мгновение, чтобы подвести итоги. Нам хочется вырваться из этой трясины, но, увы~--- мы еще больше погрязаем в~ней, и~нет спасения. 

Но разве тридцатичасовые сутки действительно смогли бы нам помочь? Разве вскоре не~стали бы мы вновь удручены той же самой проблемой, что и~при двадцатичетырех часах? Ведь заботам матери никогда не~видно конца, так же как работе студента, учителя, пастора или кого-нибудь другого, кого мы знаем. Дополнительное время нас вряд ли спасет. Детей становится больше, они взрослеют, и~времени для них требуется все больше и~больше. С~умножением профессионального и~церковного опыта усложняются наши задачи, и~увеличивается наша ответственность. Так что получается, что работаем мы все больше, а~радости это приносит все меньше.

\section*{Что же главное?}

Вдумавшись в~происходящее, мы осознаём, что причины того положения, в~котором мы находимся, заключаются не~только в~недостатке времени, истоки его глубже. На самом деле мы просто не~умеем выделить для себя главное и~заниматься сначала им. Тяжелый труд не~вредит нам. Мы все знаем, что значит долго работать с~неослабевающим напряжением, полностью отдаваясь выполнению важного дела. Охватывающие нас усталость и~утомление сменяются чувством удовлетворения и~радости от законченной работы. Если же мы оглянемся на прожитые месяцы и~годы и~вспомним о~гнетущем множестве начатых и~незавершенных дел, нас охватывает тяжкое чувство досады и~утомленности, и~виной тому не~тяжелая работа, а~сомнения и~дурные предчувствия. Нам кажется, что мы упустили что-то важное. Ветры запросов других людей вынесли корабль нашей жизни на рифы отчаяния. Мы каемся, совсем как в~грехах на исповеди: сознаем, что мы много согрешили неисполнением долга и~исполнением дел никчемных. 

Несколько лет назад я~услышал от одного умудренного опытом человека, директора хлопковой фабрики:\ <<Самая большая опасность заключается в~том, что вы позволяете срочным делам мешать выполнению важных>>. Он даже не~догадывался, насколько поразили меня эти слова. Они вновь и~вновь звучат упреком в~моей душе, заставляя опять возвращаться к~размышлениям о~том, что же я~должен делать в~первую очередь, а~что откладывать на потом. Мы разрываемся между срочным и~важным. Ведь важное требует немедленного исполнения лишь в~редких случаях. С~молитвой, чтением Библии, разговором с~неверующим другом или важной книгой можно подождать. Это срочные дела требуют внимания, их бесконечная цепь не~дает нам остановиться ни на день, ни на час.

И~уже не~скажешь:\ <<Мой дом~--- моя крепость>>, ибо он не~в~состоянии укрыть нас от срочных дел~--- телефонные звонки требовательно врываются в~наш покой и~увлекают в~пучину неотложных действий. В~тот момент нам кажется, что противостоять им невозможно, они съедают все наши силы. Но проходит время, мы понимаем обманчивость их первоочередности, при мысли об отложенных важных делах нас наполняет чувство потери. Мы с~ужасом осознаем, что стали рабами, придавленными гнетом срочных дел.

\section*{Есть ли выход?}

Есть ли выход из этого тупика? Как изменить такой образ жизни? Ответ на этот вопрос мы находим в~жизни Господа нашего. В~последнюю ночь перед смертью Христос в~Своей великой молитве, записанной в~17 главе Евангелия от Иоанна, произнес удивительные слова:\ <<Я\ldots{} совершил дело, которое Ты поручил мне исполнить>>\Footnote{1}{Иоанна 17:4}.  

Как же мог Он так сказать~---\ <<совершил>>? Его трехлетнее служение на земле казалось слишком коротким, еще так много оставалось такого, что требовало Его помощи. Блудница на пиру у~Симона получила прощение и~обрела новую жизнь, но сколько еще людей продолжало бродить по улицам, не~зная такого счастья. На каждые десять исцеленных калек были сотни остававшихся со своей болью. И~все же в~ту последнюю ночь, несмотря на множество полезных и~срочных дел, которые Господь мог бы сделать, в~душе Его царил покой~--- Он знал, что совершил дело, порученное Ему Богом.

Евангелия свидетельствуют о~том, как тяжело трудился Христос. Рассказывая об одном из Его до предела загруженных дней, Марк пишет:\ <<При наступлении же вечера, когда заходило солнце, приносили к~Нему всех больных и~бесноватых. И~весь город собрался к~дверям. И~Он исцелил многих, страдавших различными болезнями; изгнал многих бесов\ldots{}>>\Footnote{2}{Марка 1:32-34}. В~другой раз большое стечение больных и~калек заставило Его пропустить ужин и~трудиться так долго, что Его ученики стали говорить:\ <<Он вышел из Себя>>\Footnote{3}{Марка 3:21}. Однажды Иисус с~таким усердием учил собравшихся, что уснул после этого в~лодке, переплывая Геннисаретское озеро вместе с~учениками. Сон Его был столь крепок, что даже буря не~разбудила Его. Что это, как не~смертельная усталость?

\pagestyle{lheadings}

В то же время Его жизнь никогда не~была лихорадочной, у~Него всегда находилось время для людей. Он мог часами беседовать с~одним человеком, как, например, с~самаритянкой у~колодца. Вся жизнь Его была удивительно уравновешена, пронизана чувством времени. Когда Его братья хотели, чтобы Он отправился в~Иудею, Он ответил им:\ <<Мое время еще не~настало\ldots{}>>\Footnote{4}{Иоанна 7:6}. Иисус не~разрушал Своих даров излишней поспешностью. 

В~своей книге\ <<Дисциплина и~культура духовной жизни>> \ А.~И.~Вайтхем замечает:
\begin{quote}
<<У этого Человека есть цель, которая Ему по силам, внутренний покой, что привносит в~Его перегруженную до предела жизнь атмосферу беззаботности. Кроме того, Он владеет секретом и~умением обращаться с~неприятной стороной жизни~--- страданиями, разочарованиями, враждой, смертью~--- так, что дурное служит во благо, прославляя величие Бога истинного. Так и~смерть Его явилась драгоценным венцом столь внезапно оборвавшейся тридцатитрехлетней жизни и~превратила ее в~жизнь\ \glqq завершенную\grqq{}. Любуясь равновесием и~красотой этой человеческой жизни, мы не~можем не~замечать то, что сделало ее такой>>.
\end{quote} 

\section*{Ждите повеления Божьего}

В чем же все-таки секрет Его труда? Ответ на это мы находим в~Евангелии от Марка, где описан один насыщенный день Иисуса. Марк обращает внимание на то, что Иисус\ <<утром, встав весьма рано, вышел и~удалился в~пустынное место, и~там молился>>\Footnote{5}{Марка 1:35}. В~этом и~состоял секрет Его жизни и~труда~--- Иисус в~молитве склонялся перед Отцом  и~ожидал от Него повелений и~сил исполнить их. У~Него не~было все заранее расписано на каждый день Его жизни. Отец открывал Иисусу Свою волю день за днем, когда Тот обращался к~Нему с~молитвой. И~это позволяло Иисусу отодвигать срочные дела и~посвящать Себя важным.

Об этом же свидетельствует и~история Лазаря. Что могло быть важнее срочного известия, полученного Иисусом от Марии и~Марфы:\ <<Господи! Вот, кого Ты любишь, болен>>\Footnote{6}{Иоанна 11:3}? Иоанн описывает происходящее после этого словами, которые кажутся парадоксальными:\ <<Иисус же любил Марфу и~сестру ее, и~Лазаря. Когда же услышал, что он болен, то пробыл два дня на том месте, где находился>>\Footnote{7}{Иоанна 11:5-6}. В~чем была срочность? Ясно~--- не~допустить смерти возлюбленного брата. Но что было важным для Бога? Воскресить Лазаря из мертвых. Потому и~было позволено Лазарю умереть, а~позднее Господь воскресил его в~подтверждение Своих величественных слов:\ <<Я есмь воскресение и~жизнь; верующий в~Меня, если и~умрет, оживет>>\Footnote{8}{Иоанна 11:25}.

Мы можем задаваться вопросом, почему служение нашего Господа было таким коротким, разве не~могло оно продлиться еще лет пять или десять? Почему так много страдальцев остались в~своем несчастии? Писание не~дает ответов на эти вопросы, они остаются сокрытыми от нас завесой тайны~--- неисповедимы пути Господни. Но мы знаем одно: именно смиренная молитва и~ожидание повелений Божьих делали Иисуса свободным от гнета срочных дел. Благодаря этому Он обретал осознание цели, твердость шага, способность выполнить любое дело, порученное Богом. Именно поэтому в~ту последнюю ночь Иисус и~смог сказать:\ <<Я совершил дело, которое Ты поручил Мне исполнить>>.

\section*{Зависимость делает тебя свободным}

Свободу от гнета срочных дел мы находим в~примере и~обетовании нашего Господа. Под конец Своего острого спора с~фарисеями в~Иерусалиме Иисус сказал тем, кто верил в~Него:\ <<\ldots{}если пребудете в~слове Моем, то вы истинно Мои ученики, и~познаете истину, истина сделает вас свободными\ldots{} Истинно, истинно говорю вам: всякий, делающий грех, есть раб греха\ldots{} Итак, если Сын освободит вас, то истинно свободны будете>>\Footnote{9}{Иоанна 8:31-36}.

Многих из нас Христос освободил от наказания за грех. Позволим ли мы Ему освободить нас и~от гнета срочных дел? Он указывает путь:\ <<Если пребудете в~Слове Моем>>. Да, путь к~свободе лежит через молитву и~размышление над Словом Божьим. ~именно так мы познаем волю Божью.

\pagestyle{lheadings}

П.~Т.~Форсайт однажды сказал:\ <<Нет большего греха, чем безмолитвенность>>. Мы склонны думать, что к~числу худших грехов относятся убийство, прелюбодеяние, воровство. Но корень их всех в~самодостаточности, в~независимости от Бога. Забывая обращаться к~Нему в~молитве упования на Его наставления и~Его силу, мы если не~устами, так делами заявляем, что не~нуждаемся в~Нем. В~какой степени наше служение заражено отношением\ <<сам справлюсь>>?

Противоположность такой независимости~--- молитва, в~которой мы признаем необходимость Божьего наставления и~поддержки. Говоря о~нашем зависимом от Бога положении, Дональд Бейли замечает:

\begin{quote}
<<Иисус прожил земную жизнь, полностью положившись на Бога, и~мы все должны жить так же. Подобная зависимость не~разрушает человеческую личность. Напротив, человек только тогда достигает истинного и~полного самовыражения, когда живет, отдав свою жизнь Богу, всю без остатка. Только так человек может быть самим собой. Только это~--- истинная человечность>>.
\end{quote}

Без молитвы о~Божьей помощи наше служение невозможно. Как перерыв во время игры в~футбол, она дает нам возможность передохнуть и~разработать новую тактику. Уповая на Господа, мы освобождаемся от гнета срочных дел. Бог открывает нам Себя и~нас самих. Он открывает нам Свои намерения и~замыслы. Дело само по себе не~является поводом к~срочным действиям, основанием к~этому может быть только воля Божья, ибо только Ему известны пределы наших возможностей. 

 <<Как отец милует сынов, так милует Господь боящихся Его. Ибо Он знает состав наш, помнит, что мы~--- персть\Footnote{10}{пыль, земля, прах (Библейский словарь В.~П.~Вихлянцева)}>>\Footnote{11}{Псалом 102:13-14}. 
Не Бог взваливает на наши плечи непосильную ношу, не~Он доводит нас до язвы, неврозов, сердечных приступов и~инсультов. Все это~--- результат наших собственных ошибок, совершенных под давлением обстоятельств. 

\section*{Семь раз отмерь}

Если современный бизнесмен хочет преуспеть, он должен помнить: совершенно необходимо выкраивать время для подведения итогов и~планирования. В~бытность свою президентом компании\ <<Дюпон>>, г\=/н~Гринвольт сказал:\ <<Одна минута, отведенная на планирование, экономит 3\==4 минуты при исполнении>>. Многие предприятия по сбыту продукции провели потрясающие изменения в~своей работе, значительно умножив доходы. Как? Они решили послеобеденные часы каждой пятницы посвящать обсуждению и~планированию работы на следующую неделю. Если руководитель слишком занят и~не находит времени на планирование, может  произойти так, что его место вскоре займет другой, у~которого время найдется. Если христианин слишком занят и~не находит времени остановиться и~задуматься о~своей духовной жизни, забывает обращаться к~Богу в~молитве упования, то очень скоро он становится рабом срочных дел. Он может работать день и~ночь, но так и~не совершить дело, порученное ему Богом.

Если начинать день с~молитвы и~размышления, мы не~пожалеем о~затраченном времени, ибо это поможет нам вновь приблизиться к~Богу. Обдумывая ожидающие вас часы, доверьтесь Его воле. В~эти минуты покоя мысленно перечислите предстоящие вам дела в~порядке важности, не~забывая о~тех обязательствах, которые вы уже взяли на себя. Опытный генерал всегда планирует битву заранее, а~уж потом отдает солдатам приказ идти на врага. Он не~откладывает принятие главных решений до того момента, как будет открыт огонь. Но он всегда готов изменить свой первоначальный план в~случае непредвиденных обстоятельств. Итак, старайтесь осуществить все заранее намеченное на день, однако будьте готовы и~к непредвиденным обстоятельствам. Например, вдруг раздастся телефонный звонок, которого вы никак не~ждали. Возможно, устоять перед искушением согласиться на новое предложение окажется нелегко. Но если в~первый момент и~кажется, что у~вас найдется свободное время, не~спешите с~ответом, попросите пару дней на размышление. Удивительно, едва лишь смолкнет просящий голос в~трубке, как это срочное дело уже не~кажется таким неотложным.

Преодолев дурманящее чувство срочности, возникшее во время разговора, вы уже можете трезво взвесить все\ <<за>> и\ <<против>>. У~вас теперь есть время обратиться к~Господу и~понять, какова же Его воля.

Помимо ежедневного времени уединения с~Богом, отведите один час в~неделю на подведение итогов своей духовной жизни. Сядьте и~запишите все, чему Бог научил вас за это время, определите цели на будущее. Постарайтесь ежемесячно посвящать этому один день, размышляя над планами на более длительный промежуток времени. Нередко вас будут постигать неудачи. Любопытно, что чем больше у~вас дел, тем труднее вам с~ними справиться без подведения итогов и~планирования, но тем труднее и~найти свободное время для этих задач. Вы становитесь подобны фанатику, который, чувствуя, что движется не~туда, начинает бежать вдвое быстрее. Так, суетное служение Господу может стать бегством от Него. Но если вы найдете время для общения с~Богом и~подведения итогов своей духовной жизни, то обретете свежий взгляд на свою работу.

\section*{Не расслабляйтесь}

Самое сложное в~жизни христианина~--- найти достаточно времени для ежедневного общения с~Богом, для еженедельного и~ежемесячного подведения итогов, и~часто человек годами не~может заставить себя делать это. Эти часы так важны для нас именно потому, что именно тогда мы получаем наставления, повеления и~поддержку от Бога. Поэтому сатана старается сделать все возможное, чтобы помешать нам. Но мы по опыту знаем, что только таким образом можем спастись от гнета срочных дел. Сам Иисус преуспел именно благодаря этому. Он не~закончил всех срочных дел в~Палестине, которые мог бы совершить, но то дело, которое Ему поручил исполнить Бог, Он исполнил. Если мы будем уверены, что выполняем волю Божью, то отчаяние нам не~грозит. И~ничто эту уверенность не~заменит. Что может сравниться с~чудесным знанием того, что сегодня, сейчас, на этом месте мы исполняем волю Божью? Только благодаря этой уверенности мы думаем о~наших незавершенных делах со спокойствием, зная, что Бог позаботится о~них.

Не так давно трагически погиб от пуль Пол Карлсон. Несмотря на свою молодость, он уже защитил диссертацию и~мог бы преуспевать, занимаясь научными исследованиями в~пыльной библиотеке. Однако он избрал путь миссионера, который не~всегда бывает безопасным. Нелепая смерть? Но Провидением ему была уготована жизнь именно той длины, какую он прожил, и~дело, которое Бог поручил ему, он исполнил. Большинство из нас проживут дольше и~умрут в~более спокойной обстановке, но когда наступит наш час уходить, не~будет большей радости, чем та, что рождена уверенностью в~том, что дело, порученное Богом, совершено. Путь ясен: если мы постоянно пребудем в~Слове нашего Господа, то будем достойными Его учениками. И~Он освободит нас от гнета срочного для того, чтобы мы могли исполнить важное, то есть Его волю.

\thispagestyle{lheadings}

\end{multicols}

\vspace{0.3cm}
\hrule
\vspace{0.3cm}
\begin{footnotesize}
\noindent
Originally published as \textsc{``Tyranny of the Urgent''}\ by Charles~E.~Hummel.
Copyright \copyright{} 1967 InterVarsity Christian Fellowship/USA. Translated and
distributed with permission from InterVarsity Press, P.O. Box 1400,
Downers Grove, IL 60515, USA. \url{www.ivpress.com}

\vspace{0.2cm}
\noindent
Цитаты Библии в~тексте выполнены по Синодальному переводу.

\vspace{0.2cm}
\noindent
Перевод статьи: Христианское издательство <<Триада>>, 101000, Москва, а/я 371, \url{info@triad.ru}, 
\linebreak  \url{www.triad.ru}, +7~495~963-54-79, +7~495~964-85-89.

\vspace{0.2cm}
\noindent
Редактирование и~верстка: Навигаторы, Россия; \url{navigators.ru@gmail.com}
\end{footnotesize}

{\tiny 
\begin{verbatim}
$LastChangedDate: 2008-11-19 10:57:13 +0300 (Срд, 19 Ноя 2008) $
$Revision: 42 $
\end{verbatim}

}
\end{document}
