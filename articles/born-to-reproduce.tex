% $Date: 2008-11-19 10:49:58 +0300 (Срд, 19 Ноя 2008) $
% $Rev: 41 $
% $Author: Konstantin Dorichev $
% $URL: file:///var/svn/repos/condor/latex/born-to-reproduce.tex $
%
\documentclass[12pt,article,a4paper,fittopage]{ncc}
\usepackage[resetfonts]{cmap}
%\usepackage[utf-8]{inputenc}
\usepackage[russian]{babel}
%\ToCenter[h]{17cm}{25cm}
\FromMargins[h]{2cm}{2cm}{1cm}{1cm}
\usepackage{url,multicol}
\usepackage[plain,headings]{nccfancyhdr}
\newpagestyle{lheadings}[headings]{%
	\fancyhead[c]{\nouppercase{%
		\fancycenter{\slshape\rightmark}{\textit{<<Рождены, чтобы духовно воспроизводить>>~--- Д.~Тротман}}{\thepage}}}%
}
\pagestyle{lheadings}

%\pdfinfo{
%   /Author (Dawson Trotman)
%   /Title  (Born to reproduce)
%   /Subject (Spiritual multiplication)
%  /URL (http://lifestream.org/pdf/RuHLM.pdf)
%   /Keywords (Russian translation)
%   /Producer (kdorichev@gmail.com)
%}

\hyphenation{tran-sla-tion pro-du-ced writ-ten agre-ement per-mis-sion div-is-ion Navi-ga-tors Colo-rado Ori-gi-nally pub-li-shed Eng-lish copy-right res-er-ved inc-lu-ding}

\newcommand{\HRule}{\vspace{0.2cm}\noindent\rule{\linewidth}{0.1mm}}

\clubpenalty=10000
\widowpenalty=10000

\begin{document}
\thispagestyle{empty}

\begin{center}
%\includegraphics{dawson_trotman}
\vspace{0.5cm}
\textsc{\LARGE{РОЖДЕНЫ ДУХОВНО ВОСПРОИЗВОДИТЬ}}\\[0.3cm]
Доусон \textsc{Тротман}
\end{center}

\begin{multicols}{2}

\section*{Предисловие}

Летом 1955 года мне выпала честь впервые встретиться с~Доусоном Тротманом, лидером миссии Навигаторов. Он обладал в\'{и}дением того, как нужно обращать людей в~веру, но сердце мое возрадовалось не~только этому. Меня привело в~восторг, что Бог через этого человека показал, как сначала нужно завоевывать душу человека, а~затем научить его завоевывать и~обучать других, таким образом умножая служение и~обеспечивая массовый подход.

После этого я~часто встречал Навигаторов, обученных Тротманом или одним из его соратников. Как правило, это были люди, страстно борющиеся за людские души, хорошо знающие Слово Божье. Именно это отличало их как христиан.

С~самой первой моей встречи с~Доусоном Тротманом наша дружба и~сотрудничество начали стремительно развиваться. Мы проводили вместе много времени в~различных ситуациях, и~практически сразу же между нами возникла такая же привязанность, какая была у~Давида и~Ионафана.

Когда я~узнал этого человека ближе, мне открылся секрет его силы. В~самом начале христианского служения он и~еще один молодой человек ежедневно встречались для молитв в~течение шести недель, чтобы понять волю Господа. Эта сила духа и~преданность стали для него правилом всей жизни. Чтобы добиться таких успехов в~своем служении, он просыпался очень рано и~молился.

Бескорыстие мистера Тротмана проявлялось буквально во всем. Он никогда не~пытался утаить знания или факты, накопленные им на собственном опыте в~течение двадцати двух лет. Наоборот, он всегда охотно делился и~помогал нам в~создании более совершенной системы обучения в~рамках программы <<Возвращение к~Библии>>\Footnote{1}{``Back to the Bible'', прим. ред.}. В~результате был разработан Курс домашнего обучения для молодых христиан. Многие члены организации пожертвовали огромным количеством личного времени, помогая создавать этот курс, и~мистер Тротман осуществлял руководство на каждом этапе его разработки.

Вероятно одной из самых больших заслуг этого человека в~последние годы жизни была его неутомимая работа с~целью превратить данный курс изучения Библии в~реальность. Здесь произошло соединение опыта и~знаний, которое, мы верим, принесет много плода.

Мистер Тротман воссоединился с~Господом 18 июня 1956 года. Он погиб, спасая утопающего в~озере Шрун, штат Нью-Йорк. Какой символичный конец для жизни, посвященной служению другим! Один человек выразил это в~следующих словах: <<Я~считаю, что Доус повлиял на жизнь большего количества людей, чем кто-либо, кого я~знаю>>.

Работа Навигаторов продолжается под умелым руководством. Она базируется на прочном основополагающем принципе: не~<<один обучает всех>>, а~<<один обучает другого>>.
Моя жизнь теперь еще больше, чем прежде посвящена постоянному следованию этим принципам~--- запоминанию Библии и~проповеди Евангелия от человека к~человеку.
\begin{flushright}
---~Теодор Г. Эпп
\end{flushright}


\end{multicols}


\begin{multicols}{2}

Однажды, несколько лет назад, когда я~приехал в~Эдинбург в~Шотландии, я~стоял на Хай Стрит, неподалеку от замка. В~это время я~увидел, как мимо меня проходят отец и~мать с~детской коляской. Судя по их одежде, они были состоятельными и~выглядели очень счастливыми. Я~попытался заглянуть в~коляску, чтобы посмотреть на ребенка, и, увидев мой интерес, они остановились, давая мне возможность получше разглядеть маленького, розовощекого члена их семьи.

Я~смотрел на них еще некоторое время, пока они удалялись, и~думал о~том, как замечательно, что Бог  позволяет мужчине выбрать одну женщину, которая кажется ему самой прекрасной и~замечательной, и~она выбирает его из всех мужчин, которые встречаются ей на пути. Затем каждый из них отдает часть себя другому, и~Господь, в~соответствии со Своим замыслом, дает им возможность для воспроизводства! Как прекрасно, когда в~их семье рождается маленький ребенок, берущий какие-то черты от отца, и~какие-то~--- от матери, внешне похожий и~на отца, и~на мать. Каждый из них видит в~ребенке отражение того, кого он или она любит.

Когда я~увидел этого малыша, я~почувствовал тоску по своим детям, которых я~очень люблю и~которых я~уже довольно долго не~видел. Пока я~так стоял, я~заметил другую приближающуюся коляску. Она была расшатанная и~потрепанная. Очевидно, родители были бедными. Они оба были одеты очень просто и~скромно, но когда я~проявил интерес к~ребенку, они остановились и~позволили мне посмотреть на их розовощекого малыша с~ясными глазками с~той же гордостью, что и~первая пара.

Когда они пошли дальше, я~подумал: <<Бог дал этому младенцу, у~которого бедные родители, все то же, что Он дал первому. У~этого ребенка по пять пальцев на каждой руке, маленький ротик и~два глаза. Если о~нем как следует заботиться, эти ручки в~один день могут превратиться в~руки художника или музыканта>>.

А~затем мне в~голову пришла другая мысль: <<Как замечательно, что Господь не~выбрал богатых, образованных людей и~не сказал им: \glqq Вы можете иметь детей\grqq{}, а~бедным~--- \glqq Вы не~можете иметь детей\grqq{}. Эта радость дарована всем на земле>>.

Самой первой заповедью, данной человеку, было~--- <<плодитесь и~размножайтесь>>. Другими словами, человек должен был производить себе подобных. Господь не~велел Адаму и~Еве быть одухотворенными. Они уже были таковыми, созданные по образу и~подобию Его. Грех еще не~пришел в~мир. Он просто сказал: <<Размножайтесь. Я~хочу, чтобы таких как вы, по образу Моему и~подобию, было много>>.

Конечно, образ был искажен. Но у~Адама и~Евы появились дети. Они начали размножаться. Однако, пришло время, когда Господь решил уничтожить б\'{о}льшую часть их потомства. Он начал все снова, с~восьми человек. Два миллиарда живущих сегодня на земле произошли от восьми, спасшихся в~ковчеге, потому что те были плодовиты и~размножались.

\section*{Препятствия}

В~физическом плане существует лишь несколько причин, по которым люди не~могут размножаться. Одна из них~--- если человек не~женится. Если нет союза двух людей, они не~смогут дать потомство. Эту истину должны уяснить христиане и~в~отношении духовного воспроизводства. Когда человек становится одним из детей Божьих, он должен понять, что необходимо жить в~единстве с~Иисусом Христом, если он намерен привести и~других к~Спасителю.

Другой фактор, способный помешать воспроизводству,~--- болезнь или ущербность какой-либо части организма, необходимой для зачатия. В~духовном плане грех является той болезнью, которая мешает завоевывать потерянные души. Кроме того, иногда иметь детей людям мешает незрелость. В~мурости Своей Господь сделал так, чтобы маленькие дети не~могли сами иметь ребенка. Маленький мальчик сначала должен достичь достаточной зрелости, чтобы заработать на содержание, а~маленькая девочка~--- чтобы позаботиться о~малыше.

\pagestyle{lheadings}

Каждый должен родиться заново. Таково было желание Бога. Он не~хотел, чтобы человек просто жил и~умирал, был лишь ходячим трупом, которому однажды предстоит лечь в~землю. Подавляющее большинство людей знают, что есть жизнь после смерти, поэтому каждый, кто родился в~семье Божьей, должен стремиться к~тому, чтобы и~другие возрождались.

Человек возрождается, когда принимает Христа. <<А тем, которые приняли Его, верующим во имя Его, дал власть быть чадами Божьими, которые не~от крови, ни от хотения плоти, ни от хотения мужа, но от Бога родились>>\Footnote{2}{Иоанна 1:12,13}~--- это и~есть новое рождение. По замыслу Бога новые чада должны расти во Христе. Созданы все условия для того, чтобы они достигли зрелости и~затем начали множиться. Это верно не~только для богатых и~образованных, но для всех в~одинаковой мере. Каждый, родившийся в~семье Бога, должен дать потомство.

В~физическом плане, когда у~ваших детей рождаются дети, вы становитесь дедушкой и~бабушкой. Ваши родители становятся прадедушкой и~прабабушкой, а~их родители~--- прапрадедушкой и~прапрабабушкой. Так же должно быть и~в~сфере духовного.

\section*{Духовные младенцы}

Каждый раз, когда вам встречается христианин, который не~ведет других ко Христу, знайте, здесь что-то не~так. Возможно, он еще сам младенец в~духовном плане. При этом он может многое знать о~христианской доктрине и~обладать огромной информацией, слушая хорошие проповеди. Я~знаком с~многими людьми, которые могут поспорить о~премилленарном, постмилленарном и~амилленарном подходах в~богословии, знают все о~диспенсационализме, но при этом остаются духовно незрелыми. Именно о~таких говорил Павел в~послании к~Коринфянам: <<И~я~не мог говорить с~вами, братия, как с~духовными, но как с~плотскими, как с~младенцами во Христе\ldots >>\Footnote{3}{1~Коринфянам 3:1}.

Поскольку они сами были младенцами, они были незрелы, не~способны к~духовному воспроизводству. Другими словами, они не~могли помочь другим людя родиться вновь. Павел писал далее: <<Я~питал вас молоком, а~не твердою пищею, ибо вы б еще не~в~силах, да и~теперь не~в~силах, потому что вы еще плотские\Footnote{4}{или младенцы, прим. автора}. Ибо если между вами зависть, споры, разногласия, то не~плотские ли вы?\ldots>>\Footnote{5}{1~Коринфянам 3:2,3}. Я~знаком с~многими прихожанами, учителями воскресных школ, членами различных благотворительных обществ, которые могут сказать: <<А~вы слышали?>> и~передать какую-нибудь сплетню. В~глазах Бога они совершают отвратительное дело. Как ужасно, когда христианин узнает какую-то неприглядную историю и~начинает распространять ее! В~Писании говорится: <<Вот шесть, что ненавидит Господь, даже семь, что мерзость душе Его: \ldots язык лживый\ldots{}>>\Footnote{6}{Притчи 6:16,17}. Ах, эти христиане, знакомые мне мужчины и~женщины, которые позволяют лжи проникать в~нашу жизнь! <<Лжесвидетель, \ldots{} посевающий раздор между братьями>>\Footnote{7}{Притчи 6:19}, остается духовным младенцем, и~я~считаю это основной причиной, почему другие не~рождаются в~семью Господа через таких людей. Они духовно больны. С~ними не~все в~порядке. В~их жизни поселилась духовная болезнь. Они сами не~развились духовно. Между ними и~Иисусом Христом нет единства.

Но если все складывается правильно между вами и~Господом, вне зависимости от того, много или мало вы знаете по мнению окружающих с~интеллектуальной точки зрения, вы можете иметь духовных детей. Кстати, это может произойти, даже если вы сами очень молоды во Христе.

В~нашем офисе в~Колорадо Спрингс работает секретарем молодая женщина. Еще полтора года назад она состояла в~коммунистическом союзе молодежи Великобритании. Затем она услышала Билли Грэма и~приняла Иисуса Христа. Вскоре Господь через нее и~еще двух девушек, которые учатся с~ней в~одной школе искусств, обратил в~веру нескольких человек. Мы научили Патрисию и~других, а~они, в~свою очередь, научили тех, кого привели к~Иисусу Христу. Некоторые из них уже привели к~Христу других людей, которые тоже учат своих друзей. Сегодня Патрисия уже прабабушка, хотя ей самой во Христе всего год и~четыре месяца.

И~с~такими случаями мы сталкиваемся постоянно. Я~знаю одного моряка, который стал духовным прадедом, когда ему было всего четыре месяца во Христе. Он привел к~Богу моряков, которые привели к~Богу других моряков, и~те, в~свою очередь, привели к~Богу еще других моряков. При том ему самому ему было всего четыре месяца. Как так получилось? Бог использовал жизнь этих молодых христиан, как чистый канал, когда они были переполнены чувством первой любви к~Иисусу, и~из их сердец неиспорченное зерно Слова Божьего было посеяно в~сердцах других людей. Оно пустило корни. Вера пришла, когда они услышали Слово. Они родились снова через веру в~Иисуса Христа. Они смотрели на христиан, которые вели их к~Богу, и~делили с~ними радость, покой и~восторг этого пути. И,~радуясь, они стремились поделиться своим знанием с~кем-нибудь еще.

В~любом собрании христиан, я~уверен, найдутся люди, которые верят уже пять, десять или двадцать лет, но не~могут сказать, что они живут во имя Иисуса Христа, чтобы производить для Него. Кто-то скажет: <<Я~раздал сто тысяч трактатов>>. Хорошо, но сколько ты овец привел в~стадо?

\pagestyle{lheadings}

Недавно я~беседовал с~двадцатью девятью кандидатами в~миссионеры. Все они были выпускниками университетов, библейских школ или семинарий. В~качестве члена комиссии я~проводил с~ними собеседования в~течение пяти дней, с~каждым~--- от получаса до часа. Среди вопросов, которые я~задавал, два были самыми важными. Первый касался их молитвенной жизни. <<Как вы проводите время, посвященное молитвам?~--- спрашивал я~их.~--- Что вы можете сказать о~своем времени уединения с~Господом? Чувствуете ли вы, что оно проходит так, как хотелось бы Господу?>>

В~этой группе из двадцати девяти человек только один сказал: <<Я~считаю, что моя молитвенная жизнь проходит правильно, так как надо>>. Остальным я~тогда задал следующий вопрос: <<Почему ваша молитвенная жизнь проходит не~так, как должно быть?>>

Обычно ответ был таким: 

---~Видете ли, я~сейчас нахожусь здесь, на этих летних сборах. У~нас очень насыщенный курс занятий. Мы практически проходим годовой объем всего за десять дней. Мы очень заняты.

---~Хорошо,~--- говорил~я,~--- давайте вернемся к~вашей учебе в~колледже. Можете ли вы сказать, что ваша молитвенная жизнь была успешной в~то время?

---~Ну, не~совсем.

Мы старались проследить весь их путь в~вере и~приходили к~выводу, что с~тех самых пор, как они познали Спасителя, им ни разу не~удавалось добиться успеха в~молитвенной жизни. В~этом-то и~была причина их духовного бесплодия~--- в~отсутствии единения с~Иисусом Христом.

Еще я~задавал им такой вопрос:

---~Вы собираетесь в~другие страны. Вы надеетесь, что Господь через вас будет приводить к~христианской вере мужчин и~женщин. Так?

---~Да.

---~Вы хотели бы, чтобы они вели плодотворную христианскую жизнь, так? Вы не~хотите, чтобы они приняли решение, а~затем вернулись к~обычной мирской жизни, так?

---~Да.

---~Тогда позвольте задать вам следующий вопрос: сколько конкретно человек, по именам, пришли к~Иисусу Христу через вас и~живут теперь для Него?

Большинство было вынуждено признать, что они готовы пересечь океан и~выучить иностранный язык, но еще не~завоевали ни единой души, живущей теперь с~Иисусом. Некоторые из них сказали, что они многих людей убедили посещать церковь, другие сказали, что убедили некоторых взять приглашение на религиозные собрания.

---~Неужели вы надеетесь,~--- спрашивал~я,~--- что, если вы пересечете океан и~выучите иностранный язык, вы сможете сделать с~людьми, которые относятся к~вам с~подозрением, чей образ жизни вам чужд, то, что вы не~смогли сделать здесь?

Эти вопросы относятся не~только к~миссионерам или к~тем, кто собирается стать миссионерами. Это относится ко всем детям Господним. Каждый Его ребенок должен производить себе подобных.

Делаете ли это вы? Если нет, то почему? Из-за отсутствия единения с~Иисусом Христом, вашим Господом, отсутствия близости отношений, которая является частью Его великого замысла? А~может это происходит из-за какого-то греха в~вашей жизни, в котором вы не~раскаялись, и этот грех сделал вас бесплодным? А~может вы все еще являетесь духовным младенцем? <<Ибо, судя по времени, вам надлежало быть учителями; но вас снова надо учить первыми началам слова Божия\ldots{}>>\Footnote{8}{Евреям 5:12}.

\pagestyle{lheadings}

\section*{Как воспитывать способных воспроизводить}

Мы не~можем донести Благую Весть до всех уголков земли не~потому, что она недостаточно сильна.

Двадцать три года назад мы посвятили много времени тому вновь родившемуся моряку, о~котором я~упоминал. Мы показали ему, как воспитывать таких же, как он сам. На это ушло много времени, очень много времени. Это не~было простое занятие, проведенное за тридцать минут, в~спешке, с~торопливым прощанием и~приглашением заглянуть вновь на следующей неделе. Мы действительно были вместе. Мы говорили о~его проблемах, учили его не~только слышать Слово Божье и~читать Его, но и~изучать Его. Мы объясняли ему, как сделать так, чтобы его сердце стало колчаном стрел Божьего Слова, чтобы Святой Дух мог брать стрелы из его сердца, как из колчана, вкладывать ему в~уста, как в~лук, и~пронзать сердца людей во имя Иисуса Христа.

Он нашел несколько заинтересовавшихся человек на своем корабле, но никто из них не~стремился отдавать всего себя Богу. Они ходили в~церковь, но когда дело доходило до серьезной работы, они не~справлялись. Он пришел ко мне через месяц и~сказал:

---~Доусон, я~не могу убедить этих ребят на корабле заняться делом.

---~Послушай,~--- ответил я~ему,~--- ты попроси Бога послать тебе хотя бы одного такого человека. Пока не~будет хотя бы одного, не~будет и~двух. Попроси Бога послать человека, у которого будут те же устремления сердца, что и у тебя.

Он начал молиться. Однажды он пришел ко мне и~сказал: <<Кажется, я~нашел такого человека>>. Позже он привел с~собой одного юношу. Только через три месяца работы он смог найти человека с~таким же сердцем, как у~него самого. Тот первый моряк был не~из тех людей, которых нужно подталкивать и~задаривать подарками, чтобы они что-то сделали. Он любил Господа и~был готов платить настоящую цену за то, чтобы увлекать за собой других. Он по-настоящему работал с~этим новым чадом Христовым, и~вскоре оба парня начали духовно расти и~стали духовными родителями. На том корабле сто двадцать пять человек обрели Спасителя до того, как он пошел ко дну во время боя у~Пёрл Харбор. Сегодня бывшие моряки с~того боевого корабля работают миссионерами на четырех континентах. Однако, с~чего-то все должно было начаться. Самая большая уловка дъявола~--- остановить это движение  еще до того, как оно начнется. Он остановит и~вас, если вы позволите ему это сделать.

Есть христиане, которые всю жизнь как будто бегут по замкнутому кругу и~тем не~менее стремятся стать духовными родителями. Возьмем типичный пример. Вы встречаете своего знакомого утром, когда он направляется на работу, и~спрашиваете: 

---~Почему ты идешь на работу?

---~Чтобы зарабатывать деньги.

---~А~зачем ты зарабатываешь деньги?

---~Мне нужно есть и~набираться сил, чтобы работать и~зарабатывать еще больше денег.

---~А~зачем тебе еще больше денег?

---~Чтобы покупать одежду, идти в~хорошей одежде на работу и~зарабатывать еще денег.

---~А~зачем тебе еще деньги?

---~Мне нужно купить дом или платить аренду, чтобы было место, где бы я~мог хорошо отдохнуть, пойти на работу и~заработать еще больше денег.

И~так далее. Многие христиане так и~бродят кругами. Но вы продолжаете расспрашивать и~задаете вопрос:

---~А~чем ты еще занимаешься?

---~Я~нахожу время, чтобы служить Богу. Я~иногда проповедую. 

\pagestyle{lheadings}

Но за всем этим стоит одно желание~--- стать духовным отцом. Он молится о~том, чтобы Бог послал ему человека, которого он мог бы научить. Это может занять шесть месяцев. Не обязательно шесть месяцев, может быть и~меньше, но предположим, что потребуется шесть месяцев, чтобы данный ему Богом человек начал самостоятельно питать себя Божьим Словом, делиться им и научился помогать в том же другим. Тот первый в~конце шестого месяца найдет другого человека. И~каждый последующий также начнет учить другого через шесть месяцев. В~конце года их будет лишь четверо. Они могут вести занятия по Библии или помогать в~проведении уличных собраний, но их главной заботой будет являться человек, которого они обучают, и~то, как он духовно возрастает. В~конце года все четверо соберутся, помолятся вместе и~решат: <<Пусть нас ничто не~отвлекает. Будем нести Благую весть многим, но будем особенное внимание уделять хотя бы одному человеку и~постоянно поддерживать его>>.

Предположим, в~конце следующего полугода каждый найдет еще одного человека. Значит, через полгода их станет восемь. Все они будут продолжать свое дело, и~через два года их будет 16. Через три года~--- 64. Через пять лет их будет 1024. Через пятнадцать с~половиной лет таких людей станет примерно 2~147~500~000. Это сегодняшннее население земли в~возрасте старше трех лет.

Но подождите! Предположим, что после того, как первый человек А~помог Б, и~Б уже готов учить следующего, Б~сбивается с~пути, отвлекается, исчезает и~не производит последователя. Через пятнадцать с~половиной лет вместо 2~147~500~000 их будет всего лишь 1~073~750~000, только потому, что дъявол сделал Б~бесплодным.

Господь обещал Аврааму: <<\ldots в~Исааке наречется тебе семя>>\Footnote{9}{Бытие 21:12}, поэтому Авраам долго, очень долго ждал сына. Обещание Бога сделать Авраама отцом многих народов было в~том единственном сыне~--- Исааке. Если бы в~то время уже жил Гитлер, и~если бы он убил Исаака, когда Авраам занес над Исааком нож на горе Мориа, Гитлер таким образом мог бы одним ударом уничтожить всех евреев.

Я~думаю, именно поэтому сатана изо всех сил старается, чтобы христианин был постоянно занят, занят, занят и~не производил себе подобных.

Мужчина, где твой мужчина? Женщина, где твоя женщина? Где тот человек, которого ты привел к~Господу и~который теперь идет по жизни с~Господом?

В~двадцатой главе третьей книги Царств есть история о~человеке, который поручил пленника слуге и~велел ему хорошо того охранять. Но слуга был постоянно занят то там, то здесь, и~пленник сбежал.

Проклятие сегодняшнего дня заключается в~том, что мы слишком заняты. Я~не говорю о~занятости тем, чтобы заработать на хлеб. Мы слишком заняты христианской работой. В~духовной сфере мы очень активны, но при этом не~очень продуктивны. А~продуктивность приходит как результат того, что мы называем <<последующей работой>>.

\pagestyle{lheadings}

\section*{Специализация~--- духовное воспроизводство}

Пять лет назад Билли Грэм подошел ко мне и~сказал:

---~Доус, мы хотим, чтобы ты помог нам после евангелизации с~последующей работой. Я~изучал великих евангелистов и~историю евангелизационных кампаний и~не обнаружил, чтобы у~них была какая-то система поддержки новообращенных. За месяц проводимой нами кампании у~нас в среднем шесть тысяч человек решают обратиться ко Христу. Я~чувствую, что ты, с~твоим опытом работы, мог бы помочь нам.

Я~ответил:

---~Билли, я~не могу утвердить в вере шесть тысяч человек. Я~всегда работаю с~отдельными людьми или небольшими группами.

---~Послушай, Доус,~--- продолжал он,~--- где бы я~ни был, везде встречаю Навигаторов. Они были в~моей школе в~Уитоне. Они есть и~моей теперешней школе. (Он был в~то время директором Северо-Западных школ). Значит, в~этом что-то есть.

---~У~меня просто нет времени,~--- сказал~я.

Потом он вновь попытался меня уговорить. В~третий раз он уже начал умолять меня и~сказал:

---~Доус, я~просто ночами не~сплю от мыслей, что же будет с~новообращенными после завершения кампании.

Я~тогда как раз собирался в~Форм\'{о}зу\Footnote{10}{Тайвань, прим. ред.}, и~ответил:

---~Пока я~там нахожусь, я~буду об этом молиться, Билли.

Я~бродил по песчаным пляжам Формозы и~молился по два-три часа в~день:

---~Господи, как же я~смогу это сделать? Я~не успеваю сделать даже ту работу, которую Ты посылаешь мне самому. Как же я~смогу отдать полгода Билли? Но Господь уже возложил эту заботу мне на сердце.

И~почему Билли попросил именно меня заняться этим делом? Я~сказал ему в~день отъезда в~Формозу:

---~Билли, придется тебе подыскать кого-нибудь другого.

Он обнял меня за плечи и~спросил:

---~Кого другого? Кто на этом специализируется? 

Да, я~специализировался на этом вопросе.

\pagestyle{lheadings}

Что должно произойти, чтобы мы сбросили свое самодовольство и,~придя домой, стали истово молиться:  <<Господи, пошли мне юношу или девушку, которых я~мог бы привести к~Иисусу Христу, или которые уже пришли к~Нему, но еще младенцы, как христиане, чтобы я~мог их научить и~помог им самим воспроизводить в~духовном плане>>?

Как мы радуемся, когда множество людей заполняют места на наших собраниях! Но где ваш человек? Я~бы предпочел одного  живого <<Исаака>>, чем сотни мертвых, бесплодных или незрелых.

\section*{Начало последующей работы}

Однажды, много лет назад, я~ехал на своем маленьком Форде модели~Т и~вдруг увидел молодого человека, который шел по улице. Я~остановился и~предложил его подвезти. Сев в~машину, он выругался и~добавил: <<Да, не~просто тачку  тормознуть>>. Когда я~слышу как имя моего Спасителя упоминают всуе, это всегда причиняет мне боль. Я~вынул из кармана буклет, объясняющий суть Евангелия, и,~протянув ему, сказал: <<Парень, прочитай-ка вот это>>.

Он взглянул на меня и~спросил: <<Мы с~Вами случайно не~встречались раньше?>>

Я~пригляделся к~нему повнимательней. Его лицо показалось мне знакомым. Мы стали вспоминать и~выяснили, что уже встречались около года назад на этой же самой дороге. Когда я~в~тот раз предложил его подвезти, он ехал на работу в~гольф-клуб, где подносил игрокам клюшки и~мячи. И~когда он сел в~машину, он начал так же, как и~в~этот раз~--- с~упоминания имени Иисуса Христа. Я~возразил против того, как он использует святое имя, открыл Новый завет и~объяснил ему путь к~спасению. В~тот день он принял Иисуса, как своего Спасителя. На прощание я~показал ему отрывок из послания к~Филиппийцам: <<Будучи уверен в~том, что начавший в~вас доброе дело будет совершать его даже до дня Иисуса Христа>>\Footnote{11}{Филиппийцам 1:6}. <<Благослови тебя Бог, сынок, читай Библию>>,~--- сказал я~и,~радостный, отправился по своим делам.

Через год ничто в~этом парне не~напоминало о~новой жизни и~о~новом существе, как будто он никогда и~не слышал об имени Иисуса.

У~меня всегда была страсть обращать людей в~веру. Но после того как я~встретил этого парня во второй раз, я~стал возвращаться к~своим <<обращенным>>, попытался отыскать их. И,~должен вам сказать, мое сердце заныло. Казалось, отрывок из послания к~Филиппийцам не~срабатывает.

Ко мне как-то пришел мальчик, армянин, и~стал рассказывать о~тех людях, которых он обратил в~веру. Они все были армянами, и~у~него был с~собой их список в~качестве доказательства.

---~Хорошо,~--- сказал~я,~--- а~как дела вот у~этого человека?

---~С~ним дело обстоит не~очень хорошо. Он отступил от веры,~--- ответил он.

---~А~у~этого?

Мы прошлись по всему списку и~оказалось, что среди них нет ни одного, кто вел бы христианский образ жизни.

Я~сказал:

---~Дай-ка мне свою Библию.

Потом я~открыл ее на послании к~Филлиппийцам, подложил листок картона прямо под стих 1:6, вытащил из кармана нож и~нацелился на это место.

Он схватил меня за руку и~воскликнул:

---~Что Вы делаете?!

---~Я~собираюсь вырезать этот стих,~--- сказал~я.~--- Он не~работает.

А~знаете в~чем была моя ошибка? Я~рассматривал этот стих в~отрыве от контекста, который должен включать стихи с~третьего по седьмой. Павел там не~говорит: <<Ладно, Господь начал дело, Он его и~завершит>>. А~ведь именно так некоторые и~говорят; они подходят ко мне, сообщают, что обратили кого-то в~веру и~добавляют: <<Я~вверил его Богу>>.

Представьте, встречаете вы главу большой семьи и~спрашиваете:

---~Кто занимается твоими детьми?

---~Детьми, семьей? А~я~вверил их Богу.

\pagestyle{lheadings}

Сейчас я~бы сказал такому человеку:

---~У~меня есть для тебя стих из Библии: <<Если же кто о~своих и~особенно о~домашних не~печется, тот отрекся от веры и~хуже неверного>>\Footnote{12}{1~Тимофею 5:8}.

Павел говорит пресвитерам церкви в~Ефесе: <<Внимайте себе и~всему стаду, в~котором Дух Святый поставил вас блюстителями\ldots{}>>\Footnote{13}{Деяния 20:28}. Нельзя требовать от Бога, чтобы Он был блюстителем. Он сделал вас блюстителем.

Мы начали работать над программой последующей работы. Основное внимание в~течение двух или трех лет мы уделяли тому, чтобы находить новообращенных и~помогать им, а~затем нчинали с~ними работу по системе Навигаторов. В~результате мы меньше человек приводили ко Христу, но уделяли им больше времени. И~вскоре я~мог сказать так же, как Павел говорил Филиппийцам: <<Благодарю Бога моего при всяком воспоминании о~вас, всегда во всякой молитве моей за всех вас принося с~радостью молитву мою, за ваше участие в~благовествовании от первого дня даже доныне>>\Footnote{14}{Филиппийцам 1:3-5}. Павел оставался всегда со своими обращенными в~ежедневных молитвах и~в~их обучении. И~только потом смог он сказать: <<Будучи уверен в~том, что начавший в~вас доброе дело будет совершать его даже до дня Иисуса Христа>>\Footnote{15}{Филиппийцам 1:6}. И,~как бы в~продолжение этого, в~седьмом стихе говорится: <<Как и~должно мне помышлять о~всех вас, потому что я~имею вас в~сердце в~узах моих, при защищении и~утверждении благовествования, вас всех, как соучастников моих в~благодати>>.

До этого я~забывал заботиться о~тех людях, которых Бог привел к~Себе через меня. Но с~этого времени я~стал больше уделять времени тому, чтобы помогать им. Поэтому, когда тот первый моряк обратился ко мне, я~видел как важно было посвятить ему полных три месяца. Я~видел в~нем Исаака. Исаак родил Иакова, а~у~Иакова было двенадцать сыновей, и~через них произошел целый народ.

\section*{Дело Господа требует много времени}

Чтобы привести человека к~Богу, требуется от двадцати минут до двух часов. Но потребуется от двадцати недель до двух лет для того, чтобы он созрел, одержал победу над своими грехами и~возникающими в~результате проблемами. Он должен научиться принимать правильные решения. Его нужно научить опасаться разных <<-измов>>, которые могут опутать его своими щупальцами и~утащить в~сторону.

Но когда вы нашли такого человека, ваше служение удваивается, даже более чем удваивается. Знаете почему? Когда вы учите этого человека, он видит, как вы это делаете, и~начинает вас копировать. 

Если бы я~был пастором в~церкви, и~у~меня были пресвитеры, дьяконы, которые собирали бы пожертвования, члены церковного хора, я~бы сказал: <<Спасибо, Господь, за Твою помощь. Мы нуждаемся в~Тебе. Восхвалите Господа за эту дополнительную работу, которую Он вам посылает>>, но при этом я~бы продолжал настаивать на том, что считаю главным делом. Я~бы говорил: <<Плодитесь и~размножайтесь>>. Все остальное лишь сопутствует главной цели~--- обращать мужчин и~женщин к~Иисусу Христу и~затем помогать им в~их последующей христианской жизни.

Где ваш мужчина? Где ваша женщина? Существуют ли они? Просите Господа послать их вам. Загляните в~свое сердце. Спросите Господа: <<Неужели я~духовно бесплоден? Если да, то почему?>>

Не позволяйте вашему невежеству стоять у~вас на пути. В~самом начале существования Навигаторов у~нас было правило: когда мы приглашали домой на ужин моряков, в~конце вечера каждый должен был процитировать стих из Библии. Я~просил об этом следующим образом: <<Процитируйте стих, который вы запомнили за последние два дня, если есть таковой. Если нет, тогда любой стих>>. Однажды, когда мы цитировали стихи наизусть один за другим по кругу, дошла очередь до моей трехлетней дочери. Моряк, который был с~нами в~первый раз, решил, что она еще не~может наизусть цитировать Писание, и,~не дав ей возможности ничего сказать, начал читать свой стих. Она посмотрела на него, и~на ее лице было написано: <<Я~тоже человек>>. Затем она прочитала Иоанна 3:16 по-своему: <<Господь так возлюбил мил, что Он отдал Своего единолодного Сына, чтобы \textit{всякий} велующий в~Него не~погиб, но имел жизнь вечную>>. Она с~ударением произнесла слово <<всякий>>, потому что, когда учила стих, именно это слово ей не~удавалось.

Через какое-то время моряк снова пришел к~нам и~сказал мне: <<Знаете, я~ведь в~тот вечер собирался процитировать именно этот стих из Писания, потому что только его и~знал. Но на самом деле я~его не~знал, потому что когда малышка Руфь прочитала его и~сказала ``всякий'', я~подумал: ``Это же обо мне!'' Когда я~вернулся к~себе на корабль, я~принял Иисуса Христа>>. Сегодня этот молодой человек~--- миссионер в~Южной Америке.

Отец моей жены оставался неверующим несколько лет после того, как мы с~ней поженились. И~снова Господь пришел в~жаждущее сердце через детей. Когда Руфи было три года, а~Брюсу~--- пять, они поехали к~дедушке с~бабушкой в~гости. И~дедушка захотел, чтобы они почитали ему детские стишки. Он начал: <<Дженни туфлю потеряла\ldots{}>>, <<Жил--был у~бабушки серенький\ldots{}>>. Но дети смотрели на него, не~понимая, потом спросили: <<А~кто такая Дженни?>> Он решил, что они ничего не~знают.

Но мама сказала:

---~Кое-что они знают. Брюс, прочти Римлянам 3:23.

И~Брюс процитировал. Потом спросил: 

---~Еще процитировать, дедушка?

---~Ну, конечно,~--- ответил дед.

Брюс начал цитировать стихи из Писания, прочел их около пятнадцати. Время от времени вступала Руфь. Дед был в~восторге. Он повел их показывать соседям, потом тетям и~дядям, демонстрируя, как его внуки прекрасно знают Писание. А~в это время Слово Божье уже начало действовать. И~вскоре Святой Дух через голоса детей посеял Свое семя в~его сердце. <<Из уст младенцев и~грудных детей Ты устроил хвалу\ldots{}>>\Footnote{16}{Псалом 8:3}.

Те, кто обращают людей в~веру, делают это не~благодаря своим знаниям, но благодаря Тому, Кого они знают, насколько хорошо они сами Его знают, и~насколько сильно они хотят, чтобы и~другие знали Его.

<<Но я~боюсь>>,~--- говорят некоторые. Помните: <<Боязнь пред людьми ставит сеть; а~надеющийся на Господа будет безопасен>>\Footnote{17}{Притчи 29:25}. Ничто на земле, кроме греха, незрелости и~отсутствия общения с~Богом, не~сможет помешать вам духовно воспроизводить. Более того, ничто на земле не~может помешать вновь рожденному жить во Христе, если у~него есть духовные родители, заботящиеся о~нем и~дающие ему духовную пищу, которую посылает Господь, и~которая необходима для их нормального духовного роста.

Следствие подчинено причине неумолимым законом. Когда вы сеете семена Божьего Слова, вы пожинаете результаты. Не каждое сердце воспримет Слово, но те, которые воспримут, будут возрождены. Когда рождается новая душа, заботьтесь о~ней, как Павел заботился о~новых верующих. Он верил в~необходимость последующей работы. Будучи евангелистом, он был очень занят, но находил для этой работы время. Новый завет содержит много посланий Павла, которые на самом деле были письмами поддержки новообращенным.

Верил в~необходимость такой работы и~Иаков. <<Будьте же исполнители слова, а~не~слышатели только, обманывающие самих себя>>, писал Он\Footnote{18}{Иакова 1:22}. Того же мнения придерживался и~Петр: <<Как новорожденные младенцы, возлюбите чистое словесное молоко, дабы от него возрасти вам во спасение>>\Footnote{19}{1-е~Петра 2:2}. Верил в~то же и~Иоанн: <<Для меня нет большей радости, как слышать, что дети мои ходят в~истине>>\Footnote{20}{3-е Иоанна 4}. Все послания Петра, Павла, Иакова и~б\'{о}льшая часть посланий Иоанна являются пищей для молодых христиан.

Евангелие распространилось по всему миру за первые сто лет без радио, телевидения или печатной продукции, потому что эти люди плодили таких же, как они сами, которые затем также начинали воспроизводить. Но сегодня у~нас много просто просиживающих~--- людей, которые думают, что если они регулярно посещают церковь, делают солидные пожертвования и~приводят с~собой других в~церковь, они совершают свою часть работы.

Где твой мужчина? Где твоя женщина? Где твой юноша? Где твоя девушка? Каждый из нас, вне зависимости от возраста, должен стараться учить Писание наизусть. В~одной воскресной школе было две женщины. Одной из них было 72 года, а~другой~--- 78 лет, и~обе они справились с~системой Тематического запоминания Писания наизусть, которую предлагают Навигаторы. У~них было чем поделиться с~другими.

Посейте в~сердце своем Его драгоценное семя. Вы обнаружите, что Бог сам приведет вас к~тем, кого вы сможете, в~свою очередь, привести к~Иисусу Христу. Сегодня сердца многих уже готовы к~восприяию Евангелия.

\thispagestyle{lheadings}

\end{multicols}

\hrule
\vspace{0.3cm}
\begin{footnotesize}
\noindent
This translation is produced by written agreement with and permission from NavPress, a division of The~Navigators, Colorado Springs, Colorado, U.S.A.  Originally published in English as \textsc{``Born to reproduce''}\ by Dawson Trotman, copyright \copyright\ by NavPress, \url{www.navpress.com}.  All rights reserved including this translation.

\vspace{0.2cm}
\noindent
Перевод, редактирование и~верстка: Навигаторы, Россия; \url{navigators.ru@gmail.com}

\vspace{0.2cm}
\noindent
Цитаты Библии в~тексте выполнены по Синодальному переводу.

\end{footnotesize}

{\tiny 
\begin{verbatim}
$LastChangedDate: 2008-11-19 10:49:58 +0300 (Срд, 19 Ноя 2008) $
$Revision: 41 $
\end{verbatim}

}
\newpage
\thispagestyle{empty}
\section*{Вопросы для размышления}

\noindent
Каково, с точки зрения автора, Божье призвание для каждого христианина?

\HRule

\noindent
Что может помешать христианину духовно воспроизводить?

\HRule

\noindent
Запишите ниже имена людей, которых лично Вы привели ко Христу, если таковые есть, и охарактеризуйте их текущее духовное состояние:

\vspace{0.1cm}
\begin{center}
\begin{tabular}{|l|c|c|}
\hline
\hspace{2.5cm}Имя\hspace{2.5cm} & Следует ли за Христом? & Воспроизводит ли? \\
\hline
1. & & \\
\hline
2. & & \\
\hline
3. & & \\
\hline
4. & & \\
\hline
5. & & \\
\hline
6. & & \\
\hline
7. & & \\
\hline
8. & & \\
\hline
9. & & \\
\hline
10. & & \\
\hline
\end{tabular}
\end{center}

\vspace{0.1cm}
\noindent
Что вы собираетесь предпринять на этой неделе, чтобы лучше исполнять свое призвание~--- духовно воспроизводить?

\HRule

\HRule

\noindent
Заключите с Господом завет о своем решении.

\HRule

\HRule

\vspace{0.3cm}
\noindent
Дата: \rule{5cm}{0.2pt} \hspace{2cm} Ваше имя: \rule{6.5cm}{0.2pt}

\end{document}
